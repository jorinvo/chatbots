About Conversational Interfaces


Having defined the concept and character of chatbots, one apparent attribute of the technology under examination is that its domain of application involves interaction with one or more users.

Any form of interaction requires an interface as a way to interact.
An **interface** is generally described as a *"point where two systems, subjects, organizations, etc. meet and interact"* and in the area of computing it can be further defined as a *"device or program enabling a user to communicate with a computer"*. [^1]

This definition leaves a broad array of possible manifestations.
However, the range of suitable means of communication can be further narrowed by including another aspect of the previous definition, namely that interaction happens in the form of conversation.

Many communication mechanisms can be excluded by focusing on this characteristic of the interaction.
<br>
The term *conversation* strongly suggests the usage of natural, human language as a means of interaction while discouraging the idea of using an interface consisting purely of static or graphical elements.
<br>
Picturing a *conversational* interface one might expect a flow of interactions consisting of elements from all involved parties similar to the interfaces humans use to interact with each other.
<br>
It should also be noted that conversations are not limited to written language. Spoken language is also a possible interface for communication.
<br>

Focusing on written, text-based communication, it becomes apparent that not all text-based interfaces fit the characteristics of a *conversational interface*.
<br>
Classical command line interfaces often used to interact with computers are one example of text-based interfaces which don't have the attributes of conversational interfaces.
As implied in the naming, these interfaces use *commands* for interaction.
A command is an *"authoritative order"* [^2] which contrasts with a conversation.
<br>
The term *conversation* has a connotation of complex, non-linear communication where each involved party understands the underlaying ideas communicated opposed to merely receiving the characters the words consist of.

This deeper understanding of the intends a user has and the ability to adjust the interaction with a conversation consisting of customized parts sets conversational interfaces, and thus chatbots, apart from other interfaces.



[^1]: https://en.oxforddictionaries.com/definition/interface
[^2]: https://en.oxforddictionaries.com/definition/command
