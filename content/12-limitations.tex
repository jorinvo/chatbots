\section{Limitations}
\label{limitations}


While chatbots have many promising properties, chatbots also come with restrictions
which frame them as unfeasible solutions for certain scenarios.
\\

One fundamental property of the architecture of chatbot systems,
is that they are not software running on the devices of users.
\\
They mimic the nature of chat that the counterpart a user is communicating with is using a separate device.
\\
This fundamental design of chatbot implementations is similar to the basic
idea of browsing the \emph{world wide web}, whereby a network connection is a mandatory requirement.
\\
Typically network connections are part of the Internet
and such a system can be described as purely online.
\\

This underlying design has certain implications for the usage of chatbots.
\\

First and foremost this implies that all scenarios which are obliged to work without a network connection
cannot make use of chatbots.
\\
Exemplary areas of applications affected by this implication are
applications that forbid network connections due high security standards
and also products targeting rural and remote locations with unstable network connections.
\\

Another consequence is that a certain amount of latency is unavoidable with networked applications.
\\

When information needs to be transfered from one device to another,
it needs to be transported through more physical space.
\\
For the example chatbot, which is based on \emph{Facebook Messenger},
this means information needs to be transfered from the user's device to a data center,
where Facebook is operating their \emph{Messenger platform},
further to another server where the developed chatbot software is running,
and back to the user through the intermediary data center.
\\
Many factors influence the accumulated latency,
of which certain factors can hardly be influenced by the developers and operators of the chatbot software.
\\
Aspects of networking such as the capabilities of the user's devices,
the conditions the \emph{Internet service provider} of the user is operating under,
\emph{domain name lookups},
\emph{IP package routing} and associated \emph{package loss},
the locations of Facebook's data centers,
and the performance of event processing and forwarding of the \emph{Messenger platform},
these are only a selection of unknowns when operating a chatbot.
\\
The unavoidable latency and the unpredictable parties involved in the networking layout
make it clear, that chatbots are not suited for any time-critical applications.
It can not be guaranteed that a user receives a response within a unnoticeable period of milliseconds or with a delay of tens of seconds.
\\

Not only the dependency on the network is limiting the capabilities of chatbots,
but also the very idea of using \emph{chat} as a medium.
\\
While text is undoubtedly a powerful medium,
there are certain applications where other media are better suited.
\\
Many messaging platforms already enhance the communication by adding interface elements such as buttons
and by allowing multimedia content such as images and videos to be sent.
However the linear layout fundamental to text communication remains and it limits interactivity.
\\
While chat can be a very simple and efficient way of communicating and it is appropriate in many scenarios,
other use cases are address better with existing solutions such as native or browser-based applications.
Obvious candidates, which are better served with other technology are photo and video editing applications
and also 3D gaming.
\\

Due to the limitations the medium chatbot has, it can be anticipated,
that, while they address many needs of human-computer interaction,
they are not universally applicable and they are not able to replace all previous media.
\\

Much in the same way that the invention of radio has not replaced newspapers
and neither did the introduction of television replace radio,
chatbots can be seen as a new possible communication concept,
but they can coexist with previous technology in a way that each of them focuses on the area they are best suited for.

Understanding not only the merits but also the limitations of a medium is essential for deciding
whether the medium is a good fit for a use case or if another medium is a better suited solution.
\\
