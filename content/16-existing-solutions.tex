\section{Existing Solutions} \label{existing}

Before starting a new projects it is helpful to research for existing solutions that might solve the same problem.
\\
There is already a variety of existing software applications for language studying, which have very different use cases and solve different problems.

One significant separation is between software that includes content and software the user can customize to study personal content.
\\

The first segment is the most prominent. This software is intended to enable people to self-study language and, at least partially, replace physical language courses.
\\
A main reason for the prominence of this segment can be attributed to the ability to sell content.
Professionally curating a language course curriculum requires teaching expertise and is a lot of effort, and therefore content remains expensive, not only in software but also in the form of physical textbooks.
\\
One popular example from this segment is Duolingo. ``Duolingo has courses in a handful of languages.... The courses are structured in a way like games as well—you earn skill points as you complete lessons''\cite{lifehacker}.
\\
Interestingly, Duolingo recently released a chatbot\cite{topbots2} as part of their iPhone application which enables the user to have a written conversation about certain topics with a chatbot and to thereby learn the appropriate phrases for the given scenario. Although the topic and possible phrases are restricted in each scenario, this is a first example of how chatbots can be used for language studying.

% TODO: think about metioning Mondly and Eggbun here
% https://blog.mondlylanguages.com/2016/08/25/mondly-chatbot-press-release/
% http://www.tofugu.com/japanese/japanese-learning-resources-april-2017/#eggbun


The second segment consists of software which doesn't not provide users a guideline what to study but is instead intended to support users studying their own content.
\\
Most of the software that can be found in this segment is a variation of the attempt to bring traditional flashcards to digital media.
\\
One of the most established software in this segment is Anki\cite{lifehacker}, which exists for more than ten years already and provides a flexible, but also rather complex, interface to create a various kinds of studying material.
\\
Another more recent competitor is Memrise, where users get a more intuitive interface, which also includes several gamification\footnote{The Oxford Dictionaries describe gamification as ``the application of typical elements of game playing (e.g. point scoring, competition with others, rules of play) to other areas of activity''\cite{oxfordgamification}} features to make the studying process more appealing.
\\
Both mentioned products are not restricted to a field of study and users are able to add their own content. Furthermore both products also offer mechanisms for users to share content with each other, which allows users to reuse what other users created instead.
\\

For the here planned chatbot example the second segment is more fitting, since there are no resources in the current context to curate professional content.
The following implementation is an attempt at creating a software product with a conversational text interface that is in its use cases similar to products like Anki and Duolingo while using the unique features chatbots as a medium provide compared to the here mentioned products.
