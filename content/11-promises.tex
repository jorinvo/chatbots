\section{Promises}


As explained previously in \ref{presentday}, the interest in chatbots and conversational interfaces increased in recent time, because of the advancements made in the field of artificial intelligence and the popularity of messaging platforms, mobile devices, and personal assistants like Siri.
\\
The conditions are right to think about using the newly available possibilities, however, the question remaining is why we would want chatbots. What can chatbots achieve that existing solutions are not good at, both, from a user's point of view and looking at the interests companies have.
\\

From a user's point of view chatbots can be seen as a new interface to interact with computers.
Existing interfaces are not intuitive to humans. Using technology is something humans need to learn first.
With every new application one uses and every new website one visits there is a new interfaces to adapt to.
``Adjusting to a machine does not come naturally to us. With every app you need to learn how to use it. ... Conversations come naturally to us''\cite{techinasia}.
Conversation is a way of communicating humans already know how to use because they use it to interact with other humans.
If it is possible to use this communication technology to interact with machines, it should be really intuitive for humans to use it.
``The vision for a chatbot: get machines to respond to questions like a human being''\cite{techinasia}.
\\

Further, there is a trend in consumer behavior of ``outsourcing their "chores", such as driving, shopping, cleaning, food delivery, errands''\cite{chatbotbook} to companies that offer these services.
Service companies are not a new occurrence, however, in the past it took more effort to coordinate the usage of such services.
By using technology to automate many steps of the coordination process, not only the cost can be lowered, but also the friction for customers using a service is reduced significantly.
Managing and coordinating the usage of such services is a task conversational interfaces are particularly suited for because it is a scenario that profits especially from the simplicity and low friction that characterize conversational interfaces.
\\

Users can also profit from technical advantages of chatbots over to native applications and websites.
\\
Native applications need to be downloaded first which includes all resources and not just the ones we require at this moment.
Websites are a little slimmer and one only needs to download the resources required to load the current page.
But one page still contains not only content, but also layout information, styling, decorative images and in most cases also Javascript to run some additional logic in the web browser.
\\
The only thing a chatbots needs to download over the network is the content. Everything else is provided my the platform it is embedded in.
Compared to these existing solutions, ``a chatbot uses very low bandwidth''\cite{techinasia}, which can be an important advantage not only for the perceived responsiveness but especially in places where slow network connections are still common.
\\

Providing customers a more intuitive and more direct way of interacting with a company's product is already a compelling reason for a company to be interested in the new platforms, but there are additional benefits companies can draw from conversational interfaces, which are not perceivable for the users.
\\

First, ``the cost of developing a chatbot is one-third of what is required in developing a mobile app''\cite{techinasia}. This might not be the case for every single product but in general, creating a chatbot is less work than creating a mobile application, because there is no custom design required nor  is it necessary to write code for the logic controlling the user interface.
\\

Next, ``Chat apps also have higher retention and usage rates than most mobile apps''\cite{businessinsider}. Since chatbots are part of a chat application they can take advantage of being where the attention of mobile phone users already is. A chatbot can therefore potentially gain more user engagement than a competing website or mobile application.

Another aspect of the flexible nature of using natural language as an interface is that ``chatbots are able to gain invaluable data and insights on user behavior''\cite{drum}, because firstly, users have the freedom to send any kind of information and feedback and secondly, being in the context of a conversation people tend to be more talkative than they would be in a more formal environment.
Especially for companies such as media outlets or retailers being able to further profile users can be a useful assistance in tailoring personal experiences for users and targeting them with individual offers.
\\
Additional context and user data is also available on the platform itself; when interacting with a user via a chatbot on the Facebook Messenger platform all public information of the user's Facebook profile is also available for the company to user for further personalization.
\\

Lastly, the most fundamental reasons for a company to be interested in chatbots as a platform are the before mentioned popularity and ubiquity of messenger applications.
``The question brands and publishers now face is how to engage with these private social network users''\cite{drum}.
When the attention of users is shifting away from not only non-digital media but also away from other mobile applications, including traditional social networks, companies need to find a way to reach users at the place they spend most of their time at.


% APIs ``can be accessed from computers to complete real world tasks''\cite{chatbotbook}
% Already covered in previous chapter:
%``After nearly a decade of explosive growth, mobile apps have largely stopped growing''\cite{chatbotbook}
%``Social and messaging apps emerge as big winners''\cite{chatbotbook}
% ``Artificial Intelligence has gotten a lot better''\cite{chatbotbook}
% Graphic titled ``Messaging Apps Have Surpassed Social Networks''\cite{businessinsider}
