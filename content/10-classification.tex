\section{Classification}
\label{classification}


With the amount of messaging platforms opening up for chatbot development, companies have become interested in releasing their product for this new format and some developers created new products focusing solely on the chatbot market.
\\

Chatbot development is still a new market with a lot of changes happening all the time but there are some current trends in what companies are interested in creating. The direction might change soon in the future and there are probably still many undiscovered possibilities.
\\

\subsection{Categories}

One helpful classification of chatbots is categorizing them in terms of features they provide. The following categories are adapted from the article ``7 Types of Bots'' by \emph{Dotan Elharrar, a Product Manager at Microsoft AI \& Research}\cite{bottypes}.


\paragraph{Single-feature Chatbots}

A popular category of bots provide only one single feature. These bots are limited in functionality but simple to use. One example is a Facebook bot called Instant Translator\cite{instanttranslator}; in the beginning the user selects one language to translate to and all the bot does from there on is to translate all text it receives from the users language to the selected target language.


\paragraph{Proactive Chatbots}

This category are chatbots that push information to the user instead of answering questions in conversations. Hereby the user doesn't need to interact with the chatbot, but only uses it as service to receive information at certain times. One example would a service which sends the user a daily weather forcast. Another use-case is the Chatbot from the airline KLM\cite{klm}; a user can use the service to get updates and information about their flights delivered.


\paragraph{Group Chatbots}

There is a range of functionality chatbots can provide when they interact with a whole group of people instead of only a single user. These chatbots are limited to platforms which provide the necessary features to use chatbots in group conversations. A simple example for a Group Chatbot is called Roll\cite{venturebeat}; when sending a question to Roll the chatbots answers with one name of the members of the group.


\paragraph{Simplification Chatbots}

In a few cases chatbots are used to provide users with a simpler interface to complicated existing tasks which traditionally require the handling of a lot of bureaucracy. One example is a service called DoNotPay. It is advertised as "the world's first robot lawyer"\cite{oreilly} and the service helps the user with simple legal problems, such as fighting a parking ticket.


\paragraph{Entertainment Chatbots}

One of the most popular kind of chatbots are still chatbots whose functionality consists only of having conversations with users. These services don't interact with other resources apart from the conversation itself. The previous described ELIZA belongs to this category.


\paragraph{Personal Assistants}
\label{assistants}

This category consists of chatbots that combine many different features and can be seen as platforms of their own. Siri and Alex, which have been mentioned earlier, belong to this category.


\paragraph{Optimization Chatbots}

Most companies are interested in chatbots in this category. The idea is to make an existing product more accessible by creating a chabot for users to connect to the product. By taking advantage of new platforms companies like to reduce friction for customers to use the product. The currently most obvious aspect of the chatbot platforms is the ease for users to access the products. Companies like to optimize the use of their product by making it available via the conversational interfaces of chatbots.
\\

\subsection{Products}

Use cases fitting this category can be found across many different industries. The article ``100 Best Bots For Brands \& Businesses''\cite{topbots} lists examples from different industries using chatbots to optimize access to their products. Products include beauty brands such as Sephora, consumer goods like Johnnie Walker, entertainment companies including Disney and Marvel, fashion brand such as H\&M, financial services like PayPal, food delivery from stores such as Pizza Hut, E-Commerce platforms including eBay, Traveling services such as Airbnb and Expedia, Airlines like Lufthansa and British Airways and many news outlets including Washington Post, New York Times, Forbes and BCC.
\\


As apparent from the engagement of many well established companies, brands are very interested in being present on messenger platforms.
\\

While there is big interest in targeting messaging platforms as a new market, the consumer engagement and the development efforts are nothing compared to established markets such as mobile applications or the web. Most of current products only provide limited features and are mostly created to connect customers to existing products.
\\

With increasing interest, engagement and, subsequently, financial investments the arise of more sophisticated products can be expected in the future.
