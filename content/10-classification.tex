\section{Classification of Chatbots}
\label{classification}


With the increasing number of messaging platforms opening up for chatbot development,
companies have become interested in releasing their product for this new format and some companies also create new products focusing solely on the chatbot market.
\\

It is still a new, not fully formed market, but there are certain trends for what companies are interested in creating.
\\

One helpful classification of chatbots is categorizing them in terms of features they provide.
The following categories are adapted from the article ``7 Types of Bots'' by \emph{Dotan Elharrar, a Product Manager at Microsoft AI \& Research}~\cite{bottypes}.


\paragraph{Single-feature Chatbots}

A large number of chatbots provide only one single feature.
These chatbots are limited in functionality but simple to use.
One example is a Facebook chatbot called \emph{Instant Translator}~\cite{instanttranslator};
in the beginning the user selects one language to translate to and all the chatbot does from there on is to translate all text it receives to the selected target language.


\paragraph{Proactive Chatbots}

This category describes chatbots which push information to the user instead of answering questions in conversations.
Hereby the user does not need to interact with the chatbot, but only uses it as service to receive information at certain times.
One example would a service which sends the user a daily weather forecast.
Another use-case is the chatbot from the airline \emph{KLM}~\cite{klm}; users can use the service to get updates and information about their booked flights.


\paragraph{Group Chatbots}

There is a range of functionality chatbots can provide when they interact with a whole group of people instead of only a single user.
These chatbots are limited to platforms which provide the necessary features to use chatbots in group conversations.
A simple example for a \emph{group chatbot} is called \emph{Roll}~\cite{venturebeat};
when sending a question to \emph{Roll}, the chatbots answers with a random name picked from the members of the group.


\paragraph{Simplification Chatbots}

In a few cases chatbots are used to provide users with a simpler interface to complicated existing tasks which traditionally require the handling of a lot of bureaucracy.
\\
One example is a service called \emph{DoNotPay}.
It is advertised as ``the world's first robot lawyer''~\cite{oreilly} and the service helps the user with simple legal problems,
such as fighting parking tickets.


\paragraph{Entertainment Chatbots}

A popular kind of chatbots are still chatbots whose functionality consists only of having conversations with users.
These services don't interact with other resources apart from the conversation itself.
The in \ref{eliza} on page \pageref{eliza} described \emph{ELIZA} belongs to this category.


\paragraph{Personal Assistants}
\label{assistants}

This category consists of chatbots that combine many different features and can be seen as platforms of their own.
\emph{Siri} and \emph{Alexa}, which have been mentioned earlier, belong to this category.


\paragraph{Optimization Chatbots}

Chatbots in this category try to make an existing product more accessible by creating a chatbot for users to connect to the product.
\\
The difference to a \emph{simplification chatbot} is that a \emph{optimization chatbot} does is not build on an external entity such as the legal system of a state, but instead creates a product a company has full control over.
\\
By taking advantage of new platforms, companies like to reduce friction for customers to use their products.
The currently most obvious aspect of chatbot platforms is the ease for users to access products.
Companies like to optimize the use of their products by making them available via the conversational interfaces of chatbots.
\\

Use cases fitting the category of \emph{optimization chatbots} can be found across many different industries.
The article ``100 Best Bots For Brands \& Businesses''~\cite{topbots} lists examples from different industries using chatbots to optimize access to their products.
\\
Products include beauty brands such as \emph{Sephora}, consumer goods like \emph{Johnnie Walker}, entertainment companies including \emph{Disney} and \emph{Marvel}, fashion brand such as \emph{H\&M}, financial services like \emph{PayPal}, food delivery from stores such as \emph{Pizza Hut}, E-commerce platforms including \emph{eBay}, traveling services such as \emph{Airbnb} and \emph{Expedia}, airlines like \emph{Lufthansa} and \emph{British Airways} and many news outlets including \emph{Washington Post}, \emph{New York Times}, \emph{Forbes} and \emph{BCC}.
\\


As apparent from the engagement of many well established companies, brands are very interested in being present on messenger platforms.
\\

While there is growing interest in targeting messaging platforms as new markets, currently user engagement with available chatbots has not reached the popularity of other channels such as mobile applications yet,
and most chatbots created so far do not provide much functionality, but instead redirect users to their existing products.
\\

With increasing interest, engagement and, subsequently, financial investments the arise of more sophisticated products can be expected in the future.
