\chapter{Conclusion and Outlook}


- ``Text is the most socially useful communication technology. It works well in 1:1, 1:N, and M:N modes. It can be indexed and searched efficiently, even by hand. It can be translated. It can be produced and consumed at variable speeds. It is asynchronous. It can be compared, diffed, clustered, corrected, summarized and filtered algorithmically. It permits multiparty editing. It permits branching conversations, lurking, annotation, quoting, reviewing, summarizing, structured responses, exegesis, even fan fic. The breadth, scale and depth of ways people use text is unmatched by anything.''
  http://whoo.ps/2015/02/23/futures-of-text

- chat is the new command line:
  linear flow resembles command lines developers use to interact with computers,
  but a more human understandable approach.
  computing started with command lines, then GUI was invented and we forgot about text, now we have to combine them.
  but command lines are especially powerful because we can combine commands,
  this is something that could happen in the future with open platforms instead of separate chatbots.

- text should be enhanced with UI:
  conversational interfaces engage with a story as main flow
  but doesn't have to be natural language and AI.
  Haven't touched the topic of AI-powered chatbots here,
  because fundamentals of the medium remain the same without.
  AI can be used to make it more flexible and to solve more complex scenarios.
  Actually conversation as only form of interaction is hard;
  we struggle so often to talk to other humans,
  with computers we have the advantage of not being limited to only conv.
  When talking to people we often draw things on a blackboard to clarify it,
  the same way we also should assist conv with computers by augmenting it with some helpful elements.

- Text is not the best solution for everything. It's only good if you need flexibility.
  Mostly users are lazy.
  Chatbots exist for a long time but just now there are ways to make them more usable by helping users to be lazy.
  ``If something can be tapped/clicked instead of typing, they prefer that.''
  https://chatbotslife.com/dear-bot-developers-dont-get-over-hyped-374572412fda
  ``Through our journey, we have understood that the best way to build bots for businesses is through a hybrid approach – use of buttons and quick replies along with text-based queries. This approach helps both businesses and customers,''\cite{techinasia}

  ``“Adding natural language for simple domains is overkill,” says Dennis Thomas, CTO at NeuraFlash, which develops AI tools that integrate into Salesforce ... “When you have a visual medium and buttons can accomplish the task in a couple clicks (think easy re-order), open-ended natural language is not making the user’s life easier.”''
  https://chatbotsmagazine.com/does-a-bot-need-natural-language-processing-c2f76ab7ef11


- it's story telling:
  as a developer, there is no need to worry about interface design,
  still need to worry about interaction design though;
  it is a lot like telling a story to guide user through the program,
  just way less visual



- usage scenarios:
  works well for the example scenario,
  section about products show other good candidates:
  ``Another place where NLP is a big win is when the bot’s objective is focused on helping users with the discovery phase of products or shopping.''
  https://chatbotsmagazine.com/does-a-bot-need-natural-language-processing-c2f76ab7ef11
  strong for: Sales, News, Travel
  promising: Healthcare, Banking
  https://blog.botlist.co/these-2-industries-are-the-next-big-things-for-chatbots/

- Unify bots in something like Siri, Google Search or https://luka.ai/

- ``messaging as an input method''
  ``Imagine if services could even respond directly to my input''
  http://whoo.ps/2015/02/23/futures-of-text
  https://core.telegram.org/bots#inline-mode

- don't call it chatbot
  ``“Chatbot” sets bad user expectations''
  ``the term “chatbot” sets an expectation around the user experience that the technology can’t deliver.  “Chat” is a very human word. You chat with your friends. You have chat with your neighbor. Chatting has specific connotations – it’s very casual and easy. Chats meander, and you can take them any direction you want.''
  ``“Bot” is a short, memorable, and neutral.''
  ``“It’s not what you say, it’s what the other person hears.”''
  ``If business users think “chatbots” are trivial, or if they simply prefer a fancier word to refer to the function (“business process automation”) then we’re setting ourselves up for hard conversations with potential business buyers.''
  http://botnerds.com/chatbots/

