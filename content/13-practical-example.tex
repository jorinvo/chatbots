\section{Choosing a Practical Example}


Before thinking about implementation details a suitable application for a chatbot needs to be selected.
\\
As illustrated earlier, chatbots can be used to cater a wide variety of applications.
\\

Since this example application should be a demonstration of the different aspects of developing chatbots,
it should not be too simplistic in scope.
An appropriate example covers more than one of the product categories described in section \ref{classification} on page \pageref{classification},
while being, at the same time, not too technical challenging in a specific  problem domain outside of chatbot development.
\\
A service, that accepts image files and returns the same picture with a filter applied, might be an interesting and entertaining use case for a chatbot, however, it would not be an adequate example to give a general introduction to chatbot development since, even though it would a technical interesting task to solve regarding image processing, it would not illustrated much technical details in the domain of chatbot development.
\\

The here selected example application is a system for \textbf{individual language studying}.
\\
While this idea arises from a personal need for such a system and an observed shortage in the functionalities existing solutions provide,
this application also fulfills the stated requirements of a suitable example application.
\\

The language studying chatbot covers multiple of the categories defined in the section \ref{classification}.
\\
It is mainly a \emph{simplification chatbot} simplifying the task of language studying,
but the chatbot can also be classified as a \emph{proactive chatbot} because it uses proactive features to notify the user when it is time for studying.
\\

Although there is a necessity to understand parts of the problem domain of language studying,
the required domain-specific knowledge is minimal and the main communication medium remains text-based.
\\

The idea is to build a system independent from existing learning resources, that users can use to study their own vocabulary.
Since there is no need to focus on content for specific languages, the main focus remains the implementation of the chatbot features.
\\

In short, a user is able to input new vocabulary, then the system tests the user's knowledge in appropriate learning intervals.
