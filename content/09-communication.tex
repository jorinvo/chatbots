\section{Communication Mechanisms}
\label{communication}


Nowadays there are two fundamental ways a chatbot can communicate with users;
some platforms provide user interface elements, such as buttons, that can be displayed to users;
otherwise communication is done solely with natural language.
\\

Interface elements limit user input to a number of predefined actions,
while natural language has no restrictions on possible inputs.
By suggesting possible actions in the form of a list of buttons or similar,
possible interactions become obvious and simpler for the user.
Even when limiting interaction to certain actions,
the main characteristic of a chatbot remains;
the interaction is still structured in the form of a conversation,
only that the choice of input is restricted.
\\

In certain scenarios there are too many possible user inputs to fit in a fixed list.
In these cases natural language is a more appropriate input method.
\\
Custom input requires to be parsed to extract information.
As an example, when the user is asked for a date and time, and repeating times are also allowed,
a user could express a time as \emph{"every second Tuesday at 6am and 9pm"},
while in a traditional user interface this would require a non-trivial amount of interface elements.
\\

If natural language is used for communication,
it should be clearly stated what kind of input is expected,
so that the user knows which topics and which variations of input the system understands.
\\

One of the main challenges with a natural language interface is handling the non-restricted interaction
in a coherent way;
since user input is in no way limited to a single topic,
all sorts of unexpected user reactions have to be accounted for.
As a consequence of this, there are certain conversations every chatbot has to be able to handle in some way.
This includes, but is not limited to, simple smalltalk such as questions like \emph{"How are you today?"}.
\\

Most platforms allow for a combination of the discussed communication mechanisms;
predefined actions can be displayed as suggestions to the user,
while the user is also able to use natural language.
By combining both methods it can be taken advantage of the benefits of both;
there is a clear guideline for interactions and, simultaneously,
the user is free to express any possible custom input.
