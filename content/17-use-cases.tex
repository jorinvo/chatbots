\section{Definition of Use Cases}

Building on the analysis of existing solutions in \ref{existing}, features for the chatbot need to be specified.
\\
An effective method for gathering crucial features is by finding potential users
and creating usage scenarios for their individual needs.
\\

To apply this method, the fundamental problem the application is solving needs to be defined.
\\
The issue this chatbot is trying to help with is the study of individual vocabulary.
The goal is not to provide studying material in a way a language course or a textbook does,
but instead to complement these resources with a tool to study new vocabulary and phrases the learner picked up
while studying or in a different situation in every day life.
\\


\subsection{User Stories}

The following are two individuals that might possibly use the chatbot and both of them profit from it in different ways.
\\

Clara is a 22 years old American.
She moved to New York City to go to University.
Currently she's in the last year for her Bachelor degree in economic.
In University she signed up for an evening class in Mandarin.
She uses Facebook Messenger every day to talk to her friends and when a friend sent her a link
she found the chatbot.
For her the most difficult part of the studies is to write hànzì, the Chinese characters.
Now Clara uses the chatbot to write down vocabulary in hànzì during her class,
and at home she revises the new characters by going through them using the chatbot and writing the characters
down on paper.
\\

Pierre is 29 years and born in Bordaux in France.
He studied computer science.
A year ago he moved to Berlin where he found a job at a startup.
At work everyone all communication is happening in English since the team consists of people from all around the globe.
Because Pierre is not a native English speaker, he picks up new words at work almost every single day.
Since moving to Berlin Pierre also made a few German friends and he tries to pick up new words they teach him
and he also tries to remember things he sees in the supermarket.
He found the chatbot on a news website for technology products,
and since then whenever Pierre learns a new word he gets his phone from his pocket and adds the word to the chatbot.
Since the chatbot has no restrictions on what to learn, Pierre uses it to save both, German and English, vocabulary in one place.
Pierre's daily commute from and to work takes him 40 minutes each.
Now he takes advantage of this time by taking out his phone and reviewing new vocabulary he picked up the previews days.
\\


\subsection{Functional Requirements}
\label{funcreq}

The above defined user stories can be used to extract all necessary functional requirements.
\\

First, a user needs to be able to add new vocabulary.
\\
There should not be any restrictions on what to add
and vocabulary should not be limited to single words because in many cases it is more useful
to add whole phrases instead.
\\
Each vocabulary consists of the phrase the user is trying to remember
and an explanation to help the user understand what the phrase means.
\\

Next, users need a way to revise vocabulary.
\\
There should be two possible modes for revising;
one version where users can decide on their own when to go to the next phrase
and if they remembered the phrase correctly,
and a second way whereby users type out the phrase right in the messenger application.
\\
In each case the system should keep track of wheather the user new the correct solution or not.
\\

Last, it is necessary to have a means of deciding what to study next.
\\
A user should not be required to think about what to review or even when it is the right to review vocabulary.
The chatbot needs a system to decide the review time for each vocabulary,
and ideally the user is notified when vocabulary is ready to be reviewed by sending a message to the user.
\\


These three main features can be seen as a sufficient minimal viable product, MVP, for this chatbot.
\\
For demonstration purposes it is desired to keep the product as simple as possible.
\\
The knowledge that can be taken from making decisions about the implementation and walking through the process of creating the chatbot,
is mostly independent from this particular product and can be applied to the development of similar chatbot products.
