\section{Setup and Requirements}

After knowing the features the chatbot should support,
technical requirements can be extracted
and appropriate technology can be chosen for the implementation.


\subsection{Communication}

As discussed in \ref{communication},
there are two fundamental ways for communicating with a user,
interface elements and natural language.
\\

Since the previous defined features for the minimal viable product of the example chatbot are so basic,
there are not many options required and everything can be represented in an unambitious interface.
\\

The precise actions necessary to be defined are shown in the next section.
For now it should be sufficient to know that all of them can be represented as predefined actions.
\\
However guiding the user with interface elements instead of natural language
does not imply that natural language can not be used complimentary.
\\
Quite the contrary, the example chatbot is only possible by analyzing user input;
when adding new vocabulary phrases and their explanations need to be captured,
and likewise the user's guess needs to be evaluated when studying.
\\

The implementation of this chatbot demonstrates the complementing use of interface elements and natural language side by side.
\\

By relying only on simple input parsing and no advanced artificial intelligence based natural language processing techniques,
a major source of complexity of chatbot development can be avoided.
\\
While these are useful techniques that enable previous impossible use cases,
they are not necessary to explore the fundamentals and paradigms of chatbot development.
\\

\subsection{Platform}

- platform (messenger)
  - why
    - Which features are helpful for the defined use cases?
  - why not cross-platform
    - storage
    - notifications
    - no need for nlp
  - how communication works


\subsection{Server}

\subsection{Data Model}


Since a user can save new vocabulary,
there needs to be a way to store information for each user.
\\
For the studying itself, further information has to be stored.
It is necessary to keep track of correct and incorrect guesses the user does while studying
and to decide when to study next, the time of the last study needs to be tracked too.
\\

To be able to send users a notification message when their studies are ready,
the system requires a way to schedule timers for the notifications.
The timers need to be lightweight enough to be rescheduled for every user activity,
since they can affect the time the notification is scheduled for.
\\


- language (go)
- data storage (bolt)


\subsection{Architecture}

- architecture (+ Graphic)
