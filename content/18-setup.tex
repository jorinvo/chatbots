\section{Setup and Requirements}

After knowing the features the chatbot should support,
technical requirements can be extracted
and appropriate technology can be chosen for the implementation.


\subsection{Communication}

As discussed in \ref{communication} on page \pageref{communication},
there are two fundamental ways for communicating with a user,
interface elements and natural language.
\\

Since the previous defined features for the minimal viable product of the example chatbot are so basic,
there are not many options required and everything can be represented in an unambitious interface.
\\

The precise actions necessary to be defined are shown in the next section.
For now it should be sufficient to know that all of them can be represented as predefined actions.
\\
However guiding the user with interface elements instead of natural language
does not imply that natural language can not be used complimentary.
\\
Quite the contrary, the example chatbot is only possible by analyzing user input;
when adding new vocabulary phrases and their explanations need to be captured,
and likewise the user's guess needs to be evaluated when studying.
\\

The implementation of this chatbot demonstrates the complementing use of interface elements and natural language side by side.
\\

By relying only on simple input parsing and no advanced artificial intelligence based natural language processing techniques,
a major source of complexity of chatbot development can be avoided.
\\
While these are useful techniques that enable previous impossible use cases,
they are not necessary to explore the fundamentals and paradigms of chatbot development.
\\

\subsection{Platform}

An important question to answer when development a chatbot is which platform to target.
As shown in \ref{platforms} on page \pageref{platforms} there are many possible target platforms
and some are fundamentally different.
\\

Deciding for a voice-based platform like Alexa, or for SMS or a messenger platform has consequences
for all other decisions.
\\
For the case of the example chatbot, while possible, a voice-based interface is rather unsuitable since
users should be able to control the precise spelling of the vocabulary
and further, there is currently no voice-based platform available
that supports multiple languages simultaneously.
\\

SMS communication is better suited for scenarios that only need to send a low number of messages,
since, although the prices are pretty low, there are costs for sending text messages.
\\
As of writing, Twilio charges \$0.0075 for receiving and \$0.085 for sending per message when using their Global Short Message Service API in Germany\cite{twilio}.
\\
Supposed the chatbot has 1000 active users that all study 100 phrases daily,
the costs would accumulate to \$277500 of monthly expenses\footnote{1000\times100\times30\times(0.085+0.0075)=277500}.
These prices are affordable if a company is selling airplane and send tickets directly to users' phones,
but in other scenarios another messaging platform is more suited.
\\

By choosing a messaging application as target platform, there are no monthly costs to be taken care of.
Additionally there are further interface elements for interaction available than only plain text.
But there are many major existing messaging applications and the choice can be difficult.
As we saw earlier, different platforms have geographically different target markets.
If a chatbot targets the Chinese Market WeChat would be the obvious platform to choose;
likewise Japan would be targeted by using Line.
\\
In North America and Europe Facebook Messenger and Facebook's WhatsApp are currently the leading platforms.
Since as of writing WhatsApp does not provide a publicly available API, Facebook Messenger is the biggest platform one can target.
\\
By creating the first version of the example chatbot with English as an interface language,
the target markets are mainly North America, Europe and Australia and therefore Facebook Messenger would be a natural fit as target platform.
\\

As mentioned in \ref{crossplatform} on page \pageref{crossplatform}, there are also solutions to create chatbots using a framework which allows to release the chatbot to multiple platforms at the same time, but as previously noted such a framework also has drawbacks.
\\
With the limitations in mind and to keep the example as simple as possible,
it will be implemented for a single platform without abstracting the process by using third-party frameworks.
\\

Facebook Messenger is a fitting platform not only due to popularity,
but also because it offers mechanisms for interacting with chatbots via interface elements,
which will be used in the next chapter to display action buttons in the user interface.
\\

\subsubsection{Integration}
The development of a chatbot for Facebook Messenger is similar to most other platforms.
\\
When creating and configurating a chatbot, the developer registers a webhook URL.
``The concept of a WebHook is simple. A WebHook is an HTTP callback: an HTTP POST that occurs when something happens; a simple event-notification via HTTP POST.''\cite{webhook}
\\
After the setup is done,
Facebook will now send an HTTP POST request to the registered URL containing event information for every message a user sends to the chatbot.
This way the developer has complete control over how to handle each message on any machine that is reachable via a public URL.


\subsection{Server}

\subsection{Data Model}


Since a user can save new vocabulary,
there needs to be a way to store information for each user.
\\
For the studying itself, further information has to be stored.
It is necessary to keep track of correct and incorrect guesses the user does while studying
and to decide when to study next, the time of the last study needs to be tracked too.
\\

To be able to send users a notification message when their studies are ready,
the system requires a way to schedule timers for the notifications.
The timers need to be lightweight enough to be rescheduled for every user activity,
since they can affect the time the notification is scheduled for.
\\


- language (go)
- data storage (bolt)


\subsection{Architecture}

- architecture (+ Graphic)
