\section{Setup and Requirements}

After knowing the features the chatbot should support,
technical requirements can be extracted
and appropriate technology has to be chosen for the implementation.


\subsection{Communication}

- what interactions are necessary?
- possible actions are really limited
- use of combination to cancel + other options
- use of combination to provide both input methods in parallel


\subsection{Platform}

- platform (messenger)
  - why
    - Which features are helpful for the defined use cases?
  - why not cross-platform
    - storage
    - notifications
    - no need for nlp
  - how communication works


\subsection{Server}

Since a user can save new vocabulary,
there needs to be a way to store information for each user.
\\
For the studying itself, further information has to be stored.
It is necessary to keep track of correct and incorrect guesses the user does while studying
and to decide when to study next, the time of the last study needs to be tracked too.
\\

To be able to send users a notification message when their studies are ready,
the system requires a way to schedule timers for the notifications.
The timers need to be lightweight enough to be rescheduled for every user activity,
since they can affect the time the notification is scheduled for.
\\


- language (go)
- data storage (bolt)


\subsection{Architecture}

- architecture (+ Graphic)
