\section{Use Cases and Requirements}


Building on the analysis of existing solutions in \ref{existing} on page \pageref{existing}, features for the chatbot need to be specified.
An effective method for gathering crucial features is to find potential users
and create usage scenarios for their individual needs.
\\

To apply this method, the fundamental problem the application is solving needs to be defined first.
\\
The issue this chatbot is trying to help with is the study of individual vocabulary.
The goal is not to provide studying material in a way a language course or a textbook does,
but instead to complement these resources with a tool to study new vocabulary and phrases students pick up
while studying or in different situations in every day life.
\\


\subsection{User Stories}

The following are two hypothetic scenarios of individuals that might use the chatbot and both profit from it in different ways.
\\

Clara is a 22 years old American.
She moved to New York City to go to university.
Currently she is in the last year of her bachelor degree in economics.
In university she signed up for an evening class in Mandarin.
She uses \emph{Facebook Messenger} every day to talk to her friends and she discovered the chatbot when a friend sent her a link.
For her the most difficult part of the studies is to write hànzì, the Chinese characters.
Now Clara uses the chatbot to write down vocabulary in hànzì during her class,
and at home she revises the new characters by going through them using the chatbot and writing the characters
down on paper.
\\

Pierre is 29 years and born in Bordeaux, France.
He studied computer science and a year ago he moved to Berlin where he found a job in a startup.
At work all communication is done in English since the team consists of people from all around the globe.
Because Pierre is not a native English speaker, he picks up new words at work almost every single day.
Since moving to Berlin Pierre also made a few German friends and he tries to pick up new words they teach him.
He found the chatbot on a news website for technology products,
and since then whenever Pierre learns a new word he grabs his phone from his pocket and adds the word to the chatbot.
Since the chatbot has no restrictions on what to learn, Pierre uses it to save both, German and English, vocabulary in one place.
Pierre's daily commute from and to work takes 40 minutes twice a day.
Now he uses his commute time to take out his phone and review new vocabulary he picked up the previous days.
\\


\subsection{Functional Requirements}
\label{funcreq}

All necessary functional requirements can be extracted from the above defined user stories.
\\
First, a user needs to be able to add new vocabulary.
There should not be any restrictions on what can be added
and vocabulary should not be limited to single words, because in many cases it is more helpful
to add whole phrases instead.
Each vocabulary consists of the phrase the user tries to memorize
and an explanation to help understanding the meaning of the phrase.
\\
Next, the chatbot should provide a way to revise vocabulary.
There should be two possible modes for revising;
one where users can click a button to tell whether they remembered the phrase correctly or not,
and a second mode whereby users type out the phrase themselves.
In each case the system should keep track of whether users knew the correct solution or not.
\\
Lastly, it is necessary to determine what to study next.
A user should not be required to think about what or when to review vocabulary.
The chatbot needs a system to decide the review time for each vocabulary,
and ideally the user is notified when vocabulary is ready to be reviewed by sending a message to the user.
\\
These three main features can be seen as a sufficient \emph{minimal viable product}, or \emph{MVP}.
\\

For demonstration purposes it is desired to keep the product as simple as possible.
The knowledge that can be taken from making decisions about the implementation and walking through the process of creating the chatbot,
is mostly independent from this particular product and can be applied to the development of other chatbot products.


\subsection{Non-functional Requirements}

Since this is a simple example, non-functional requirements remain minimal.
\\
\emph{Availability} of the service is not a priority, but chatbot software can be scaled similar to other software,
and redundancy can be used to ensure availability.
Since messaging platforms act as intermediary between users and the chatbot software, most platforms also re-send missed messages in case the chatbot is unavailable.
That the platform ensures availability, further lessens the priority to address it in the chatbot software itself.
\\
Similarly, \emph{security} is not a main focus here, because the messaging platform itself already handles certain security-sensitive functionality such as authentication and encryption of communication. A production scenario, though, would require further care for securing the service.
\\
\emph{Performance} is equally not a major concern.
Because the scope of the example application is limited,
the domain specific logic remains inexpensive in computation.
The main performance bottleneck is the in \ref{limitations} on page \pageref{limitations} mentioned aspect of networking and involved unknown parties.
Employing performance-improving solutions for networking issues won't be a part of the example chatbot,
but performance can be improved by choosing geographically strategic located data centers for deploying the chatbot software.
\\
A more central requirement is \emph{reusability}.
Although the example focuses on solving a specific task,
the software architecture should be designed in a way,
that appropriate parts can be reused for other chatbots in the future.
To ensure reusability the software should be \emph{documented}, \emph{stable} and \emph{extensible}.
\\
\emph{Usability} can be seen as the most important non-functional requirement.
The focus of developing the example chatbot is to design a good user experience and to explore how interface and interaction design can be best accomplished with the given medium.
