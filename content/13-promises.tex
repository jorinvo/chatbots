\section{Promises}


As explained previously in \ref{presentday}, the interest in chatbots and conversational interfaces increased in recent time, because of the advancements made in the field of artificial intelligence and the popularity of messaging platforms, mobile devices, and personal assistants like Siri.
\\
The conditions are right to think about using the newly available possibilities, however, the question remaining is why we would want chatbots. What can chatbots achieve that existing solutions are not good at, both, from a user's point of view and looking at the interests companies have.
\\

Technical:
``a chatbot uses very low bandwidth''\cite{techinasia}
APIs ``can be accessed from computers to complete real world tasks''\cite{chatbotbook}

User:
``The vision for a chatbot: get machines to respond to questions like a human being''\cite{techinasia}
``Adjusting to a machine does not come naturally to us. With every app you need to learn how to use it. ... Conversations come naturally to us''\cite{techinasia}
``outsourcing their “chores”, such as driving, shopping, cleaning, food delivery, errands''\cite{chatbotbook}

Companies:
``the cost of developing a chatbot is one-third of what is required in developing a mobile app''\cite{techinasia}
``The question brands and publishers now face is how to engage with these private social network users''\cite{drum}
``chatbots are able to gain invaluable data and insights on user behavior''\cite{drum}
``Chat apps also have higher retention and usage rates than most mobile apps''\cite{businessinsider}

% Already covered in previous chapter
%``After nearly a decade of explosive growth, mobile apps have largely stopped growing''\cite{chatbotbook}
%``Social and messaging apps emerge as big winners''\cite{chatbotbook}
% ``Artificial Intelligence has gotten a lot better''\cite{chatbotbook}
% Graphic titled ``Messaging Apps Have Surpassed Social Networks''\cite{businessinsider}
