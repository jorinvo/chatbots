\section{Definition Chatbot}
\label{defchatbot}


The term \emph{chatbot} consists of two other terms - \emph{chat} and \emph{bot}.
The meaning can be better understood by examining the two components separately.
\\

The Oxford Dictionary defines \textbf{chat} as ``an informal conversation'' and more specifically as ``the online exchange of messages in real time with one or more simultaneous users of a computer network''~\cite{oxfordchat}.
As apparent in this definition, conversations play a central role in \emph{chat} and therefore \emph{chatbots}.
Other noteworthy aspects of this definition are the inherent \emph{informal} format of a \emph{chat},
and the traits of being \emph{online} and \emph{real time}.
\\
Informality does not have to be seen as a strict requirement; however a chat message and, for example, a classical letter have different degrees of formality.
\\
Being \emph{online} and thereby not bound to a specific geographic location, device or other physicality can be seen as critical foundation for determining potential types of systems suitable for such media.
\\
The aspect of limiting communication to \emph{real time} implies restrictions on possible interactions and sets a baseline for the expected user experience.
This also excludes the usage of certain technologies which do not support the desired responsiveness.
\\

A \emph{conversation} is defined as ``a talk, especially an informal one, between two or more people, in which news and ideas are exchanged''~\cite{oxfordconversation}.
Fundamental to this definition is that there are always at least \emph{two} parties involved in communication and that information is \emph{exchanged}.
Keeping that in mind, the kind of systems involved in this should always receive and provide information;
chatbots can not work with solely unidirectional interaction.
\\

\textbf{Bot} is defined as being ``(chiefly in science fiction) a robot'' with the specific characteristics of representing ``an autonomous program on a network (especially the Internet) which can interact with systems or users, especially one designed to behave like a player in some video games''~\cite{oxfordbot}.
\\
Foremost this provides the information that \emph{bots}, including \emph{chatbots}, are \emph{programs}.
The creation of a chatbot implies the creation of an artifact in the form of a computer program.
\\
Furthermore the aspect of \emph{autonomy} and the communication over a \emph{network} can be connected with the previous described trait of a \emph{chat} to be \emph{online}.
\\
The program is given autonomy by not being bound to any specific device.
Building on this allows for different solutions than a scenario where the user is in full control of a program's behavior.
\\
Lastly there is a hint in this definition pointing out that a \emph{bot} can often be seen as a \emph{player} in a \emph{game}.
This trend towards \emph{game-like} mechanisms and the previous mentioned \emph{informality} suggest the utilization of playful interactions.
\\

Concluding from the combination of these definitions, a chatbot can be defined as an autonomous computer program that interacts with users or systems online and in real time in the form of, often play-like and informal, conversations.
