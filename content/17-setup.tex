\section{Technical Setup}


After knowing the features the chatbot should support,
technical requirements can be extracted
and appropriate technology chosen for the implementation.


\subsection{Communication}

As discussed in \ref{communication} on page \pageref{communication},
there are two fundamental ways for communicating with a user,
interface elements and natural language.
\\

Since the previous defined features for the minimal viable product of the example chatbot require not many options,
everything can be represented with unambitious interface elements.
\\

The concrete implementation of available actions is shown in the next section.
For now it should be sufficient to know that all of them can be represented as predefined actions.
\\
However guiding the user with interface elements instead of natural language
does not imply that natural language can not be used complimentary.
\\
Quite the contrary, the example chatbot is only possible by analyzing user input;
when adding new vocabulary, phrases and their explanations need to be captured,
and likewise when studying the user's guesses need to be evaluated.
\\

The implementation of this chatbot demonstrates the complementing use of interface elements and natural language side by side.
\\

By relying only on simple input parsing and no advanced natural language processing techniques,
a major source of complexity of chatbot development can be avoided.
\\
While there are useful techniques that enable previous impossible use cases,
they are not necessary to explore the fundamentals and paradigms of chatbot development.
\\

\subsection{Platform Selection}

An important question to answer when developing a chatbot is which platform to target.
As shown in \ref{platforms} on page \pageref{platforms} there are many possible target platforms
and some of them are fundamentally different to others.
\\

Deciding for a voice-based platform like \emph{Alexa}, for \emph{SMS} or for a messenger platform has consequences
for all further decisions.
\\
For the example case, a voice-based interface is less appropriate since
users should be able to control the precise spelling of the vocabulary
and further, there is currently no voice-based platform available
that supports multiple languages simultaneously.
\\

\emph{SMS} communication is better suited for scenarios that only need to send a low number of messages,
since there are costs for sending text messages.
\\
As of writing, \emph{Twilio}, a cloud communications service, charges \$0,0075 for receiving and \$0,085 for sending a message when using their \emph{Global Short Message Service API} in Germany~\cite{twilio}.
\\
Supposed the chatbot has 1000 active users that all study 100 phrases daily,
the costs would accumulate to \$277.500 of monthly expenses\footnote{$1000\times100\times30\times(0,085+0,0075)=277.500$}.
These prices are affordable if a company is selling airplane tickets or something similar,
but in other scenarios another messaging platform is better suited.
\\

By choosing a messaging application as target platform, there are no monthly costs to be taken care of.
Additionally there are further interface elements than only plain text available for interaction.
But there are many major existing messaging applications and deciding for one can be difficult.
\\
As shown in \ref{geography} on page \pageref{geography}, different platforms have geographically different target markets.
If a chatbot targets the Chinese market \emph{WeChat} would be the obvious platform to choose;
likewise Japan would be targeted by using\emph{ Line}.
\\
In North America and Europe \emph{Facebook Messenger} and \emph{Facebook's WhatsApp} are currently the leading platforms.
Since as of writing \emph{WhatsApp} does not provide publicly available access to the platform,
\emph{Facebook Messenger} is the biggest platform one can currently target.
\\
By creating the first version of the example chatbot with English as an interface language,
the target markets are primarily North America, Europe and Australia
and therefore \emph{Facebook Messenger} is a natural fit as target platform.
\\

As mentioned in \ref{crossplatform} on page \pageref{crossplatform},
there are also solutions to create chatbots using a framework that allows to release a chatbot to multiple platforms at the same time,
but as previously noted such a framework also has drawbacks.
\\
With the limitations in mind and with the goal of keeping the example as simple as possible,
it will be implemented for a single platform without abstracting the process by using third-party frameworks.
\\

\emph{Facebook Messenger} is a fitting platform not only due to popularity,
but also because it offers mechanisms for interacting with chatbots via interface elements,
which will be used in the next chapter to display buttons in the user interface.
\\

The registration and configuration of a chatbot for \emph{Facebook Messenger} is not explained in detail here,
since Facebook's online documentation already covers detailed explanations
and the majority of the settings are specific to each chatbot.
\\
It should be noted that Facebook refers to a chatbot in \emph{Messenger} simply as \emph{bot},
and to create a \emph{Messenger bot} it is required to first create a \emph{Facebook Page} and an \emph{application} for the page.
Afterwards \emph{Messenger} can be added as a \emph{product} to the application.
\\
Further information can be found in the developer documentation.\footnote{https://developers.facebook.com/products/messenger}

\subsubsection{Integration}

The development of a chatbot for \emph{Facebook Messenger} is similar to most other platforms.
\\
When creating and configuring a chatbot, the developer registers a \emph{WebHook}.
``The concept of a WebHook is simple. A WebHook is an HTTP callback: an HTTP POST that occurs when something happens; a simple event-notification via HTTP POST''~\cite{webhook}.
\\
After the setup is done,
Facebook will now send \emph{HTTP POST} requests to the registered URL containing event information for every message a user sends to the chatbot.
\\
This way the developer has complete control over handling each message on an arbitrary machine that is reachable via a public URL.


\subsection{Server Technology}

At this point that the interaction with the platform is decided, the application can be created on a custom server,
and it has to be decided how to structure the application running on this server.


\subsubsection{Programming Language}

Since all interaction with the platform happens via HTTP,
there is no restriction on which programming language to use as long as it can be accessed through HTTP.
\\
Chatbots can be written in any programming language.
\\
Depending on the specific application, different programming languages are better suited than others though.
\\

In the example case there are not many specific technological requirements.
\\
However, to be able to send notifications to users when their studies are ready,
the system requires a way to schedule timers.
The timers need to be lightweight enough to be re-scheduled for every user activity
that can affect the time the notification is scheduled for.
\\
Single-threaded programming languages, such as Python, Ruby or PHP,
mostly use separate worker processes to handle scheduled jobs,
but in this case there would need to be a worker process for each user
waiting until it is time to send notifications to the specific user.
\\
Another way to handle this in single-threaded programming languages is by using a system for \emph{asynchronous, evented I/O},
such as the \emph{asyncio} module for \emph{Python}~\cite{asyncio} or the \emph{Node.js} JavaScript runtime~\cite{nodejs}.
\\

The example chatbot is created with the \emph{Go} programming language~\cite{golang}.
\\
Go is a multi-threaded language which allows for taking advantage of all available CPUs
on a machine without the overhead of creating new processes.
\\
It is well suited for scheduling notifications and Go includes a robust web server in the \emph{standard library} which can be exposed to the public Internet without a proxy server, which is a common approach with the previous mentioned programming languages.
\\
With this setup there are less moving parts to take care of and the example can be a simple, self-contained application.
\\


\subsubsection{Data Storage}

In the example case data needs to be stored and it should be stored locally on the same machine
to keep the application simple.
\\

Since a user can save new vocabulary,
there needs to be a way to store information for each user.
\\
For the studying itself, further information has to be stored.
It is necessary to keep track of correct and incorrect guesses, it needs to be decided when to study next and the time of the last study needs to be tracked too.
\\

To send notifications additional information has to be stored.
\\
It is necessary to know when the user was last active and if the user saw the last notification,
since notifications should only be send when the user is not already active at this moment
and when the user has already seen the last notification.
\\

All of the required information is always bound to a single user.
Users can never share information with each other.
There are no further relations in the data.
\\
Without relational data the features that relational databases such as \emph{MySQL} or \emph{PostgreSQL} provide are not used;
instead a simple key-value store can be sufficient for storing the data.
\\
The example uses an embedded key-value store called \emph{BoltDB}~\cite{boltdb}, which does not need a separate process and saves data on disk in a single file.
\\


These technical decisions are the base for the following description of the chatbot implementation.
