\section{Present Day}
\label{presentday}


With more than sixty years of history the concept of chatbots is not new.
\\
People have been fascinated with the idea of being able to \emph{talk to computers} since a long time; but past attempts have mostly been simple experiments or applications focused on the aspect of entertainment associated with science fiction inspired machines.

Lately the technology industry and press are increasingly interested in the topic of conversational interfaces.
\\
The reinforced interest can be explained by observing recent developments of technology and recent market trends.

Since Apple's release of Siri in 2011\cite{iphonelaunch} customers have become more aware of the possibilities of conversational interfaces. Even though the capabilities were limited at that time, functionality improved quickly in the following years, which can be attributed to the new competition Apple triggered in the market.

Simultaneously, in this period of time artificial intelligence gained new traction due to the success of using artificial neural networks for machine learning\cite{mltrend}.
\\
The concept of artificial neural networks ``dates back to the 1950s, and many of the key algorithmic breakthroughs occurred in the 1980s and 1990s''\cite{airevolution}, but only now they are successfully applied. This is mainly due to the increased computing power available now. A second important condition is the amount of data available today; big Internet companies specialize on collecting data, originally intended to better target advertisement, but now they can use their data to train neural networks. Neural networks have their name because they are modeled after human brains; instead of specific rules of what the program should do the machine learns from examples in similar ways to how humans learn. Some task that are too complex to solve with a rule-based program, can now be solved by collecting enough example data and letting the machine figure out the solution instead.
\\
This techniques can also be applied in the field of natural language processing, which is essential to understanding and generating text for conversational interfaces.
\\

With these new technical possibilities more people see \emph{conversation as an interface} not only as an idea of science fiction movies but instead as something that could be possible in the real world.
\\

In addition to these new technical possibilities recent market trends make chatbots more compelling. ``Computing is rapidly shifting to mobile devices''\cite{mobileusage} and ``messaging apps have surpassed social networks in monthly active users''\cite{convtrends}. This development means that users don't have space for complex interfaces on the small screens of their devices, they need a solution which is light-weight in data consumption, and they can not use complicated keyboard shortcuts. At the same time users already spend the majority of their time in messaging applications and therefore, they are well accustomed to this way of communicating.
\\

Originating from the current state of technology the increased interest in conversational interfaces creates new platforms, products and applications for chatbots.



