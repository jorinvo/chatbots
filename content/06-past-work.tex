\section{Past Work}


Before exploring new technology one should examine prior work and learn from past ideas, both, succeed and also failed attempts.
\\
This section presents a selection of events from the last century, which introduced ideas forming the present definition of chatbots.
\\
It should be noted that this is not an attempt to give an all-encompassing overview about the history of computing,
instead the aim is to explain where the concept of chatbots and the interest of creating them originated from.


\subsection{The Turing Test}

Even before the term \emph{chatbot} was coined people started working on machines that interact with humans through natural language.
\\
A first milestone was the 1950 paper \emph{Computing Machinery and Intelligence} by Alan Turing~\cite{turing}.
The ideas he formulated back then are still fundamental to the concept of a \emph{chatbot} in todays world
and his thoughts are still central to many discussions about artificial intelligence.
\\
The most famous idea from this paper is the so called \emph{Turing Test},
which is meant to decide whether a machine posses human-like intelligence or not.
\\
Originally Turing called the test \emph{imitation game} whereas the experiment consists of a human interacting with two parties via \emph{textual messages}.
One of the parties is another human and one is a machine.
The test subject does not know upfront which party is a machine and which one is a human, but only that one of them will be a machine.
During the \emph{game} the human can interact with the other party
via what is nowadays called \emph{chatting},
and is free to use any variation of messages.
\\
If the human is not able to tell which of the two parties is a machine and which one is a human, the machine passes the \emph{Turing Test}.
\\

When creating a chatbot or another kind of artificial intelligence this test can still be applied to test the \emph{human-likeliness} of the created machinery.
\\
While Alan Turing did not invent the first chatbot, the \emph{Turing Test} was a crucial motivation for further work
and up to this point in time the test still remains to be challenged by new systems.


\subsection{ELIZA}
\label{eliza}

Fourteen years after the Turing Test was defined, Joseph Weizenbaum started working on what would be known as the first program to pass a limited version of the Turing Test.
Joseph Weizenbaum began working at \emph{MIT Artificial Intelligence Laboratory} in 1964 and he released the \emph{ELIZA} program in 1966.
\\
\emph{ELIZA} can be seen as the first known creation of a chatbot.
\\

The original version of \emph{ELIZA} was written in a programming language called \emph{MAD-Slip},
which was also created by Joseph Weizenbaum himself and it ran on the \emph{IBM 704} computer.
\\
\emph{ELIZA} creates responses to natural language messages a user inputs via a text-based terminal.
\\

The most famous implementation of \emph{ELIZA} is called \emph{DOCTOR} and simulates a \emph{Rogerian psychotherapist}.
\\
Rogerian psychotherapy is a person-centered therapy intended to let the client realize their own attitudes and behavior.
Although relying on mostly simple methods, it remains a popular treatment.
Most answers the therapist gives are questions for further details about information which the client mentioned previously.
Furthermore clients mostly keep the assumption that a therapist has specific intentions even when asking non-obvious questions.
\\

\emph{ELIZA} takes advantage of the structure of the English language;
the program takes apart sentences via pattern matching and keywords, and reuses phrases after substituting certain words.
\\
For example, a client's answer ``Well, my boyfriend made me come here.'' can be transformed to ``Your boyfriend made you come here?''~\cite{elizatest}.
\\
Certain signal words and also sentences containing no signals words can be answered with generic, static phrases.
Detecting the signal word ``alike'' in the sentence ``Men are all alike.'' \emph{ELIZA} could pick the programmed phrase ``In what way?'' as answer~\cite{elizatest}.
\\

Knowing about the nature of Rogerian psychotherapy, Joseph Weizenbaum created \emph{ELIZA} initially intended as a parody to demonstrate the simple behavior necessary for imitating this therapy.
\\
He was surprised that even people that know about the inner workings of the program ended up having serious conversations with \emph{ELIZA}.
In one anecdote Joseph Weizenbaum tells how his secretary, after starting a conversation with \emph{ELIZA,} asked him: ``Would you mind leaving the room, please?''~\cite[5]{weizenbaum}.
\\

Led by the success of the experiment he published the book \emph{Computer power and human reason: from judgment to calculation} in 1976,
which presents his thoughts about artificial intelligence,
including the differences between machines and humans and the limits of computer intelligence.
\\
In the book he admits that he had not realized ``that extremely short exposures to a relatively simple computer program could induce powerful delusional thinking in quite normal people''~\cite{bbcnowthen}.
\\
This idea coined the term \emph{Eliza Effect} which describes people assuming computers to behave like humans. This term is still in use today.


\subsection{After 1970}

Another famous program was published by the psychiatrist Kenneth Colby in 1972.
\\
He created \emph{PARRY} as an attempt to simulate a human with paranoid schizophrenia.
The implementation of \emph{PARRY} is far more complex than \emph{ELIZA},
but it also models a personality including concepts of how to have conversations.
\\
The most famous demonstration of \emph{PARRY} was at the first \emph{International Conference on Computer Communications} (ICCC) in 1972 where \emph{PARRY} and \emph{ELIZA} had a conversation with each other~\cite{internethistory}.
\\
Later on in scientific experiments \emph{PARRY} also passed a version of the \emph{Turing Test}.
\\

Further programs that have been created to pass the \emph{Turing Test} and which gained the public's attention include
\\
\emph{Jabberwacky} which was started in 1988 and attempts to learn from the user's input~\cite{jabberwacky},
\\
\emph{Dr. Sbaitso} which was released in 1991 as an \emph{ELIZA}-like demonstration for a sound card and was one of the first chatbots for MS-DOS based personal computers~\cite{pcmag},
\\
and \emph{A.L.I.C.E.}, which has been released in 1995 and became famous for its realistic behavior, that is based on heuristic patterns instead of static rules~\cite{approximatinglife}.
\\

The origin of the term \emph{chatbot} itself can be seen in a paper called ``ChatterBots, TinyMuds, and the Turing Test: Entering the Loebner Prize Competition'' published by Michael L. Mauldin in 1994, whereby \emph{chatbot} can be seen as a variation of the original term \emph{ChatterBots}~\cite{aiconf}.
\\

Up until today the \emph{Turing Test} has only been passed limited to certain domains and there is no chatbot yet that is able to simulate general human behavior indistinguishably from a real human being.
\\
It needs to be noted that, although creating an as human-like as possible system remains a popular challenge, not all applications of chatbots profit from this type of behavior.
Many systems are instead optimized to provide quick and efficient interactions and behave accordingly without attempting to hide their artificiality.
