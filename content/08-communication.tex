\section{Communication}

Nowadays there are two fundamental ways a chatbot can communicate with users;
some platforms provide user interface elements, such as buttons, that can be used;
otherwise communication is done solely with natural language.
\\

Interface elements limit user input to a number of predefined actions,
while natural language has no restrictions on possible inputs.
\\
By suggesting possible actions in the form of a list of buttons or something similar,
the interaction can be more clear for the user and thereby simpler.
\\
It should be noted that even by limiting interaction to predefined actions,
the main characteristics of a chatbot remain;
the interaction is still structure in the form of a conversation.
\\

In certain scenarios there are too many possible user inputs to fit them in a list.
Then the input has to be given as natural language.
\\
Parsing custom input can be helpful to extract information from complex user inputs.
As an example, when the user is asked for a date and time but the situation also allows for repeated times,
a user could state something like "every second Tuesday at 6am and 9pm",
while in a traditional user interface that would require a non-trivial amount of input fields.
\\
If natural language is used for communication,
it should be clearly stated what kind of input is expected
since the user is not able to know which topics and variations of input a system understands.
\\
One of the main challenges with a natural language interface is handling the non-restricted interacted
in a coherent way;
since user input is in no way limited to a certain for of expected input for the current topic,
all sorts of unexpected user reactions have to be taken care of.
\\
As a consequence of this, there are certain conversations every chatbot has to be able to handle in some way.
These include, but are not limited to, simple smalltalk such as questions like "How are you today?".
\\

Most platforms allow for a combination of the discussed communication mechanisms;
predefined actions can be displayed as suggestions to the user,
while the user is still able to use custom natural language.
\\
By combining both methods it can be taken advantage of the benefits of both;
there is a clear guideline for interactions and simultaneously,
the user is free to use any possible custom input.
\\


