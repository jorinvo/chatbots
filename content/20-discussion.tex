\chapter{Discussion}

To conclude the exploration of chatbots, a more opinionated take on the merits of chatbots
and also ideas in which direction their role could evolve in the future are given here
based on thoughts from different people active in this field.
\\

To begin with it should be clarified that text is a great medium for communication as this quote~\cite{futuretext} illustrates:

\begin{quote}
Text is the most socially useful communication technology. It works well in 1:1, 1:N, and M:N modes. It can be indexed and searched efficiently, even by hand. It can be translated. It can be produced and consumed at variable speeds. It is asynchronous. It can be compared, diffed, clustered, corrected, summarized and filtered algorithmically. It permits multiparty editing. It permits branching conversations, lurking, annotation, quoting, reviewing, summarizing, structured responses, exegesis, even fan fic. The breadth, scale and depth of ways people use text is unmatched by anything.
\end{quote}

But no matter how great text is, users prefer to not to type out everything in text.
``If something can be tapped/clicked instead of typing, they prefer that''~\cite{chatbotslife}.
\\
The idea of chatbots exists for a long time already,
but just now messenger platforms add features that help developers to make chatbots more user-friendly by not requiring users to type out every response.
\\
``Through our journey, we have understood that the best way to build bots for businesses is through a hybrid approach – use of buttons and quick replies along with text-based queries. This approach helps both businesses and customers,'' says Mounish, a co-founder of the bot builder platforms MindIQ~\cite{techinasia}.
Neither is natural language processing based on the latest advancements in machine learning and artificial intelligence a solution for every problem~\cite{neednlp}:

\begin{quote}
“Adding natural language for simple domains is overkill,” says Dennis Thomas, CTO at NeuraFlash, which develops AI tools that integrate into Salesforce ... “When you have a visual medium and buttons can accomplish the task in a couple clicks (think easy re-order), open-ended natural language is not making the user’s life easier.”
\end{quote}

The recent additions of capabilities to messenger platforms give chatbots the ease of use other applications provide,
while chatbots can also take advantage of the flexibility and power of text as a medium.
\\

Text and natural language enable new kinds of products, where the focus is more on flexible and individual behavior than accuracy.
\\
A ``place where NLP is a big win is when the bot’s objective is focused on helping users with the discovery phase of products or shopping.''~\cite{neednlp}
Businesses selling goods, newspapers that need to spread their content and travel agencies helping customers to create individual journeys,
these are some branches which are currently particularly interested in the possibilities of natural language.
\\
Two areas the usage of chatbots is promising for are \emph{health care} and \emph{banking}~\cite{botlist}
because these fields involve a big amount of customer service of which many common situation can potentially be automated with machine-based customer support.
\\

Another interesting development, apart from enhancements of the interfaces and natural language, is the in \ref{assistants} on page \pageref{assistants} mentioned category of \emph{personal assistants}.
\\

Many chatbots, including the created example chatbot, are designed to only solve a specific problem.
They always have to find a way to handle attempts of the user to use them for tasks outside of their area of expertise.
\emph{Personal assistants} do not have these limitations
and systems like \emph{Siri} or \emph{Google Assistant} have the potential to evolve into assistants as imagined in science fiction movies and literature~\cite{assistant}.
\\

Particularly interesting are ideas for open systems where third-party developers can add additional capabilities,
and the \emph{personal assistant} has a mechanism of delegating specific tasks to the third-party systems.
An example is the development of \emph{skills} for Amazon's \emph{Alexa}~\cite{alexa}.
\\

Lastly, after focusing on \emph{chatbots} for this whole time,
it can be questioned if this is the right terminology after all~\cite{botnerds}.

\begin{quote}
First, the term “chatbot” sets an expectation around the user experience that the technology can’t deliver.
“Chat” is a very human word.
You chat with your friends.
You have chat with your neighbor.
Chatting has specific connotations – it’s very casual and easy.
Chats meander, and you can take them any direction you want.
...
If business users think “chatbots” are trivial, or if they simply prefer a fancier word to refer to the function (“business process automation”) then we’re setting ourselves up for hard conversations with potential business buyers.
\end{quote}

Depending on the circumstances a different term can be more appropriate.
\\

But no matter whether they are called \emph{chatbots}, \emph{bots}, \emph{business process automation} or \emph{conversational interfaces},
they can be used to imagine new product ideas and they bring the power and convenience of text to where users feel most at home.

