\section{Existing Solutions}
\label{existing}


When starting a new project, it is helpful to research existing solutions which solve similar problems.
There are already a variety of existing software applications for language studying, which have various use cases and solve different problems.
\\
One significant separation is between software that includes content and software users can utilize and customize to study personal content.
\\

The first segment is the most prominent.
This software is intended to enable people to self-study language and, at least partially, replace physical language courses.
A main reason for the prominence of this segment is the ability to sell content.
Professionally curating the curriculum for a language course requires teaching expertise and takes a lot of effort.
Because of this, good content for language studying remains expensive, not only in software but also in the form of physical textbooks.
One popular example from this segment is \emph{Duolingo}. ``Duolingo has courses in a handful of languages.... The courses are structured in a way like games as well - you earn skill points as you complete lessons''~\cite{lifehacker}.
\\
Interestingly, \emph{Duolingo} recently released a chatbot~\cite{topbots2} as part of their \emph{iPhone} application which enables the user to have a text-based conversation about certain topics with a chatbot to learn the appropriate phrases for the given scenario.
Although the topics and possible phrases are restricted in each scenario, this is a first example of how chatbots can be used for language studying.
\\

The second segment consists of software which does not provide users a guideline for what to study, but is instead intended to support users studying their own vocabulary.
Most of software belonging to this segment are attempts at bringing traditional flashcards to digital media.
\\
One of the most established applications in the second segment is \emph{Anki}~\cite{lifehacker},
which exists for more than ten years already and provides a flexible,
but also rather complex, interface to create various kinds of studying material.
Another more recent competitor is \emph{Memrise}, which gives users a more intuitive interface, that also includes several gamification\footnote{The Oxford Dictionaries describe gamification as ``the application of typical elements of game playing (e.g. point scoring, competition with others, rules of play) to other areas of activity''~\cite{oxfordgamification}} features to make the studying process more appealing.
Both mentioned products are not restricted to a field of study and users are able to add their own content.
Furthermore both products also offer mechanisms for users to share content with others, which allows users to reuse what other users created.
\\

For the in this work planned chatbot example the second segment is more fitting, since there are no resources in the current context to curate professional content.
\\

The following implementation is an attempt at creating a software product with a conversational text interface that is in its use cases similar to products like \emph{Anki} and \emph{Duolingo} while making use of the unique features the medium \emph{chatbot} provides.
