\section{Limitations}


In the previous section different properties of using a chatbot as medium
and solutions for certain scenarios based on those properties have been mentioned.
\\

There are further properties that affect the design of chatbot based systems,
and which can in some cases even frame certain systems as unfeasible.
\\

One fundamental property of the implementation of chatbots,
is that they are not software running on the devices of users.
They mimic the nature of other chat situations where the counterpart
a user is communicating is not on the same device as the user is using.
\\
This fundamental design of chatbot implementations is similar to the basic
idea of browsing the world wide web, whereby a network connection is a mandatory requirement.
Typically the network connections are part of the Internet
and such a system can be described as a purely online system.
\\
This underlying design has certain implications for the usage of chatbots.
\\

First and foremost this implies that all usage scenarios which are obliged to work without any network connection
cannot make use of chatbots.
Exemplary areas of applications affected by this requirement are
applications that forbid network connections due high security standards
and also products specializing in usability in rural, remote location with unstable network connections.
\\

Another implication is the unavoidable latency of networked applications.
\\
When information needs to be transfered from one device to another,
it needs to be transported through a larger physical.
\\
For the example chatbot, which is based on Facebook Messenger,
this means information needs to be transfered from the user's device to a data center,
where Facebook is operating their Messenger platform,
further to another server where the developed chatbot software is running,
and back to the user through the intermediary data center from Facebook.
\\
Many factors influence the accumulated latency,
of which certain factors can hardly be influenced by the developers and operators of the chatbot software.
\\
Aspects of networking such as the capabilities of the user's devices,
the conditions the Internet service provider of the user is operating under,
domain name lookups,
IP package routing and associated package loss,
the locations of Facebook's data centers,
and the performance of event processing and forwarding of the Messenger platform,
these are only a selection of unknowns when operating a chatbot.
\\
The unavoidable latency and the unpredictable parties involved in the networking layout
make it clear, that chatbots are not suited for any time-critical applications,
and it can not be guaranteed that a user receives within a unnoticeable period of milliseconds or with a delay of tens of seconds.
\\

Not only the dependency on the network is limiting the capabilities of chatbots,
but also the very idea of using chat as a medium.
\\
While text is undoubtedly a powerful medium,
there are certain applications where other media are better suited.
Many platforms already enhance the communication by adding interface elements such as buttons
and by allowing multimedia content such as images and videos to be sent.
However the linear layout fundamental to text communication remains and limits interactivity.
While this can be a very simple and efficient way of communicating for many scenarios,
other use cases are address better with existing solutions such as native or browser-based applications.
Obvious candidates, which are better served with other media are for example photo and video editing applications
or also 3D gaming.
\\

Due to the limitations the medium chatbot has, it can be anticipated,
that, while they can address many needs for humans to communicate with computers,
they are not universally applicable and they are not able to replace all previous media.
\\
Much in the same way that the invention of radio has not replaced newspapers
and neither did the introduction of television replace radio,
chatbots can be seen as a new possible communication concept,
but they can coexist with previous technology in a way that each of them focuses on the area they are best suited for.
