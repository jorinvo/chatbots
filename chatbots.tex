% \documentclass[12pt, openright]{report}
\documentclass[12pt, headinclude, parskip=half+, numbers=noenddot, openright]{scrreprt}
% PDF-Kompression
\pdfminorversion=5
\pdfobjcompresslevel=1
% Allgemeines
\usepackage[automark]{scrpage2} % Kopf- und Fußzeilen
% \usepackage{amsmath,marvosym} % Mathesachen
\usepackage[T1]{fontenc} % Ligaturen, richtige Umlaute im PDF
\usepackage[utf8]{inputenc}% UTF8-Kodierung für Umlaute usw
% Schriften
\usepackage{setspace} % Zeilenabstand
\usepackage{titlesec}
\onehalfspacing % 1,5 Zeilen
% Schriften-Größen
\setkomafont{chapter}{\Large\rmfamily} % Überschrift der Ebene
\setkomafont{section}{\large\rmfamily}
\setkomafont{subsection}{\large\rmfamily}
\setkomafont{subsubsection}{\normalsize\rmfamily}
\setkomafont{chapterentry}{\large\rmfamily} % Überschrift der Ebene in Inhaltsverzeichnis
\setkomafont{descriptionlabel}{\bfseries\rmfamily} % für description Umgebungen
\setkomafont{captionlabel}{\small\bfseries}
\setkomafont{caption}{\small}
% Language: English
\usepackage[english]{babel}
% PDF
\usepackage[final]{microtype} % mikrotypographische Optimierungen
\usepackage{float}
\usepackage{url} % ermögliche Links (URLs)
\usepackage{pdflscape} % einzelne Seiten drehen können
\usepackage{hyperref}
% Tabellen
\usepackage{multirow} % Tabellen-Zellen über mehrere Zeilen
\usepackage{multicol} % mehre Spalten auf eine Seite
\usepackage{tabularx} % Für Tabellen mit vorgegeben Größen
\usepackage{longtable} % Tabellen über mehrere Seiten
\usepackage{array}
%  Bibliographie
% Sort by order of appearance in document
\usepackage[sorting=none]{biblatex}
% Bilder
\usepackage{graphicx} % Bilder
\usepackage{color} % Farben
\graphicspath{{images/}}
\DeclareGraphicsExtensions{.pdf,.png,.jpg} % bevorzuge pdf-Dateien
\usepackage{subcaption}  % mehrere Abbildungen nebeneinander/übereinander
\usepackage[all]{hypcap} % Beim Klicken auf Links zum Bild und nicht zu Caption gehen
% Bildunterschrift
\setcapindent{0em} % kein Einrücken der Caption von Figures und Tabellen
\setcapwidth{0.9\textwidth} % Breite der Caption nur 90% der Textbreite, damit sie sich vom restlichen Text abhebt
\setlength{\parskip}{0pt}
\setlength{\abovecaptionskip}{0.2cm} % Abstand der zwischen Bild- und Bildunterschrift
% Quellcode
% für Formatierung in Quelltexten, hier im Anhang
\usepackage{listings}
\definecolor{grau}{gray}{0.25}
\lstset{
	extendedchars=true,
	basicstyle=\normalsize\ttfamily,
	tabsize=2,
	keywordstyle=\textbf,
	commentstyle=\color{grau},
	stringstyle=\textit,
	numbers=left,
	numberstyle=\tiny,
	% für schönen Zeilenumbruch
	breakautoindent  = true,
	breakindent      = 2em,
	breaklines       = true,
	postbreak        = ,
	prebreak         = \raisebox{-.8ex}[0ex][0ex]{\Righttorque},
}
% linksbündige Fußboten
\deffootnote{1.5em}{1em}{\makebox[1.5em][l]{\thefootnotemark}}

\typearea{14} % typearea berechnet einen sinnvollen Satzspiegel (das heißt die Seitenränder) siehe auch http://www.ctan.org/pkg/typearea. Diese Berechnung befindet sich am Schluss, damit die Einstellungen oben berücksichtigt werden

\usepackage{scrhack} % Vermeidung einer Warnung



\addbibresource{references.bib}



\begin{document}



\pagenumbering{Roman}
\pagestyle{empty} % remove header and footer for first page

% title page
\clearscrheadings\clearscrplain
\begin{center}

HTW Berlin\\
University of Applied Sciences

\vspace{2cm}

Bachelor's Thesis

\vspace{1cm}

\begin{Large}
  Chatbots: Development and Applications
  \vspace{3cm}
\end{Large}

\begin{tabular}{rl}
{\bfseries Author} & Jorin Vogel\\
{\bfseries First Examiner} & Prof. Dr. Gefei Zhang\\
{\bfseries Second Examiner} & Prof. Dr. Carsten Busch\\
\end{tabular}

\vspace{3cm}

International Media and Computing\\
Faculty IV

\vspace{6cm}

July 29, 2017

\end{center}

\clearpage

% TODO: think about adding an abstract and a licensing notice


\tableofcontents
% \listoffigures
% \listoftables

\pagestyle{useheadings} % enable header and footer for other pages
\pagenumbering{arabic}


% \chapter{Introduction}

\pagestyle{useheadings} % enable header and footer for other pages
\pagenumbering{arabic}


% What's the problem to be solved?
This work gives a general introduction to chatbots
by explaining what they are, what they can be used for and how to develop them.
\\
No previous domain-specific knowledge is required.
\\

% Why is it a problem?
Lately as of writing topics around chatbots have received increasing attention from media and also numerous investments from different actors in the industry.
\\
At the same time not many potential users know about the existence of chatbots or about areas in which chatbots could be helpful assistance.
The topic is equally unknown to developers.
\\
While the term \emph{chatbot} is commonly used in media, the meaning mostly remains ambiguous.
\\
There is a need for further explanation of what chatbots are and further analysis to identify well suited applications for chatbots.
\\
Additionally to spreading knowledge about the potentials of chatbots and their use cases,
more developers should be enabled to create new, innovative chatbots.
\\

% How do I intent to solve it?
The lack of knowledge can be solved by providing answers to the questions of what chatbots are, what benefits they bring and how to create them.
\\
An appropriate definition of chatbots can be given by analyzing the fundamental meaning of the term \emph{chatbot} and by exploring past and current applications.
\\
Use cases of chatbots can be identified by exploring existing applications. Additionally, market trends and attributes of media and technologies can be analyzed to find new potential scenarios for the usage of chatbots.
\\
Development is best explained by creating a real chatbot and by using it to present the general principles of the development process.
\\

% How will this help the world?
Explaining what chatbots are, demystifying what to use them for and presenting how to create them,
will help more people to be able to use and create chatbots, and thereby, accelerate the development of the chatbot ecosystem.
\\
Innovation in technology and the creation of new solutions can help automating and simplifying more tasks,
which gives people the opportunity to focus on more interesting issues and accomplish more things.
\\
Chatbots have the potential to simplify and automate many existing tasks and thereby accelerate the overall technological progress.
\\

% How is this work structured?
The structure of this work follows the three main questions.
\\
To begin with, terminology is defined and applications are explored to form a definition and understanding of what chatbots are.
\\
Afterwards use cases of chatbots are identified not only through the collection of existing examples, but also through the exploration of future potentials by analyzing attributes of the relevant technologies.
\\
The second half of the work is a case study for the development of a chatbot.
The presented example guides through the process of designing user interactions for a chatbot, and additionally explains architectural decisions and technological choices, which provide a basis for other developers to build on when creating new chatbots in the future.

\chapter{Fundamentals}

Before working on a topic it is necessary to have common definitions of its vocabulary. This chapter will help aligning previous assumptions.

\section{Definition Chatbot}
\label{defchatbot}


The term \emph{chatbot} consists of two other terms - \emph{chat} and \emph{bot}.
The meaning can be better understood by examining the two components separately.
\\

The Oxford Dictionary defines \textbf{chat} as ``an informal conversation'' and more specifically as ``the online exchange of messages in real time with one or more simultaneous users of a computer network''~\cite{oxfordchat}.
As apparent in this definition, conversations play a central role in \emph{chat} and therefore \emph{chatbots}.
Other noteworthy aspects of this definition are the inherent \emph{informal} format of a \emph{chat},
and the traits of being \emph{online} and \emph{real time}.
\\
Informality does not have to be seen as a strict requirement; however a chat message and, for example, a classical letter have different degrees of formality.
\\
Being \emph{online} and thereby not bound to a specific geographic location, device or other physicality can be seen as critical foundation for determining potential types of systems suitable for such media.
\\
The aspect of limiting communication to \emph{real time} implies restrictions on possible interactions and sets a baseline for the expected user experience.
This also excludes the usage of certain technologies which do not support the desired responsiveness.
\\

A \emph{conversation} is defined as ``a talk, especially an informal one, between two or more people, in which news and ideas are exchanged''~\cite{oxfordconversation}.
Fundamental to this definition is that there are always at least \emph{two} parties involved in communication and that information is \emph{exchanged}.
Keeping that in mind, the kind of systems involved in this should always receive and provide information;
chatbots can not work with solely unidirectional interaction.
\\

\textbf{Bot} is defined as being ``(chiefly in science fiction) a robot'' with the specific characteristics of representing ``an autonomous program on a network (especially the Internet) which can interact with systems or users, especially one designed to behave like a player in some video games''~\cite{oxfordbot}.
\\
Foremost this provides the information that \emph{bots}, including \emph{chatbots}, are \emph{programs}.
The creation of a chatbot implies the creation of an artifact in the form of a computer program.
\\
Furthermore the aspect of \emph{autonomy} and the communication over a \emph{network} can be connected with the previous described trait of a \emph{chat} to be \emph{online}.
\\
The program is given autonomy by not being bound to any specific device.
Building on this allows for different solutions than a scenario where the user is in full control of a program's behavior.
\\
Lastly there is a hint in this definition pointing out that a \emph{bot} can often be seen as a \emph{player} in a \emph{game}.
This trend towards \emph{game-like} mechanisms and the previous mentioned \emph{informality} suggest the utilization of playful interactions.
\\

Concluding from the combination of these definitions, a chatbot can be defined as an autonomous computer program that interacts with users or systems online and in real time in the form of, often play-like and informal, conversations.

\section{Conversational Interfaces}


Having defined the concept and character of chatbots, one apparent attribute of the technology under examination is that its domain of application involves interaction with one or more users.
\\

Any form of interaction requires an \emph{interface} as a way to interact.
An \textbf{interface} is generally described as a ``point where two systems, subjects, organizations, etc. meet and interact'' and in the area of computing it can be further defined as a ``device or program enabling a user to communicate with a computer''~\cite{oxfordinterface}.
\\

This definition leaves a broad array of possible manifestations.
However, the range of suitable means of communication can be further narrowed by including another aspect of the previous definition, namely that interaction happens in the form of \emph{conversation}.
\\

Many communication mechanisms can be excluded by focusing on this characteristic of the interaction.
\\
The term \emph{conversation} strongly suggests the usage of natural human language as a means of interaction
while discouraging the idea of using an interface consisting purely of static or graphical elements.
\\
Further conversations are not limited to written language.
Spoken language is also a possible interface for communication.
\\

Focusing on written, text-based communication, it becomes apparent that not all text-based interfaces fit the characteristics of a \emph{conversational interface}.
\\
Classical command line interfaces, which are often used to interact with computers, are one example of text-based interfaces that do not have the attributes of conversational interfaces.
As implied in the naming, these interfaces use \emph{commands} for interaction. A \textbf{command} is an ``authoritative order''~\cite{oxfordcommand} which contrasts with a conversation.
\\
The term \emph{conversation} has a connotation of complex, non-linear communication where each involved party understands the underlaying ideas communicated opposed to merely receiving the characters the words consist of.
\\

Deeper understanding of the intends a user has and the ability to adjust the interaction with a conversation consisting of customized parts sets conversational interfaces, and thus chatbots, apart from other text-based interfaces.

\chapter{Applications}


This chapter explores different applications of chatbots.
\\
Starting with past work and looking at the present in subsequent sections,
different approaches and products are presented.
\\
Furthermore todays ecosystem is portrayed,
which includes an overview about available platforms,
existing products
and current approaches for the creation of chatbots.
\\
Moreover current and potential users of chatbot products are analyzed
and important questions about the current state of chatbot usage are addressed.
\\
Lastly the aspired goals, potentials and promises of chatbots are further identified.

% TODO: Too many sections starting with "P"
\section{Past Work}


Before exploring new technology one should know about prior work and learn from past ideas - both, succeed and also failed attempts.
\\
This section presents a selection of events from the last century, which introduced ideas forming the present definition of chatbot.
\\
It should be noted that this is not an attempt to present an all-encompassing overview about the history of computing,
instead the aim is to explain where the concept of chatbots and the interest of creating them originated from.


\subsection{The Turing Test}

Even before the term \emph{chatbot} was coined people started working on machines that interact with humans.
\\
One important milestones was the 1950 paper \emph{Computing Machinery and Intelligence} by Alan Turing\cite{turing}.
The ideas he had back then are still fundamental to the concept of a \emph{chatbot} in todays world
and his thoughts are still central to many discussions about artificial intelligence.
\\
The most famous idea from this paper is the so called \emph{Turing Test},
which is meant to decide whether a machine posses human-like intelligence or not.
\\
Originally Turing called the test \emph{imitation game} whereas the experiment consists of a human interacting with two parties via \emph{textual messages}.
One of the parties is another human and one is a machine.
The test subject does not know upfront which party is a machine and which one is a human, but only that one of them will be a machine.
During the \emph{game} the human can interact with the other party
via what is nowadays called \emph{chatting},
and is free to use any variation of messages.
\\
If the human is not able to tell which of the two parties is a machine and which one is a human, the machine passes the \emph{Turing Test}.
\\

When creating a chatbot or another kind of artificial intelligence this test can still be applied to test the \emph{human-likeliness} of the created machinery.
\\
Up to this point in time the \emph{Turing Test} still remains to be challenged by new systems.


\subsection{ELIZA}

Fourteen years after the Turing Test was defined, Joseph Weizenbaum started working on what would be known as the first program to pass a limited version of the Turing Test.
Joseph Weizenbaum began working at \emph{MIT Artificial Intelligence Laboratory} in 1964 and he released the \emph{ELIZA} program in 1966.
\\

The original version of \emph{ELIZA} was written in a programming programming language called \emph{MAD-Slip},
which was also created by Joseph Weizenbaum himself and it ran on the \emph{IBM 704} computer.
\\
\emph{ELIZA} creates responses to natural language messages a user inputs via a text-based terminal.
\\

The most famous implementation of \emph{ELIZA} is called \emph{DOCTOR} and simulates a \emph{Rogerian psychotherapist}.
\\
Rogerian psychotherapy is a person-centered therapy intended to let the client realize their own attitudes and behavior.
Relying on mostly simple methods, it remains a popular treatment.
Most answers the therapist gives are questions for further details about information which the client mentioned previously.
Furthermore clients mostly keep the assumption that a therapist has specific intentions even when asking non-obvious questions.
\\

\emph{ELIZA} takes advantage of the structure of the English language;
the program takes apart sentences via pattern matching and keywords and reuses phrases after substituting certain words.
\\
For example, a client's answer ``Well, my boyfriend made me come here.'' can be transformed to ``Your boyfriend made you come here?''\cite{elizatest}.
\\
Certain signal words and also sentences containing no signals words can be answered with generic, static phrases.
Detecting the signal word ``alike'' in the sentence ``Men are all alike.'' \emph{ELIZA} could pick the programmed phrase ``In what way?'' as answer\cite{elizatest}.
\\

Knowing about the nature of Rogerian psychotherapy, Joseph Weizenbaum created \emph{ELIZA} initially intended as a parody to demonstrate the simple behavior necessary for imitating this therapy.
\\
He was surprised that even people that know about the inner workings of the program ended up having serious conversations with \emph{ELIZA}.
In one anecdote Joseph Weizenbaum tells how his secretary, after starting a conversation with \emph{ELIZA,} asked him: ``Would you mind leaving the room, please?''\cite[5]{weizenbaum}.
\\

Led by the success of the experiment he published the book \emph{Computer power and human reason: from judgment to calculation} in 1976,
which presents his thoughts about artificial intelligence,
including the differences between machines and humans and the limits of computer intelligence.
\\
In the book he admits that he had not realized ``that extremely short exposures to a relatively simple computer program could induce powerful delusional thinking in quite normal people''\cite{bbcnowthen}.
\\
This idea coined the term \emph{Eliza Effect} which describes people assuming computers to behave like humans. This term is still in use today.


\subsection{After 1970}

Another famous program was published by the psychiatrist Kenneth Colby in 1972.
\\
He created \emph{PARRY} as an attempt to simulate a human with paranoid schizophrenia.
The implementation of \emph{PARRY} is far more complex than \emph{ELIZA},
but it also models a personality including concepts of how to have conversations.
\\
The most famous demonstration of \emph{PARRY} was at the first \emph{International Conference on Computer Communications} (ICCC) in 1972 where \emph{PARRY} and \emph{ELIZA} had a conversation with each other\cite{internethistory}.
\\
Later on in scientific experiments \emph{PARRY} also passed a version of the \emph{Turing Test}.
\\

Further programs that have been created to pass the \emph{Turing Test} and which gained the public's attention include
\\
\emph{Jabberwacky} which was started in 1988 and attempts to learn from the user's input\cite{jabberwacky},
\\
\emph{Dr. Sbaitso} which was released in 1991 as an \emph{ELIZA}-like demonstration for a sound card and was one of the first chatbots for MS-DOS based personal computers\cite{pcmag},
\\
and \emph{A.L.I.C.E.}, which has been first released in 1995 and became famous for its realistic behavior, that is based on heuristic patterns instead of static rules\cite{approximatinglife}.
\\

The origin of the term \emph{chatbot} itself can be seen in a paper called ``ChatterBots, TinyMuds, and the Turing Test: Entering the Loebner Prize Competition'' published by Michael L. Mauldin in 1994, whereby \emph{chatbot} can be seen as a variation of the original term \emph{ChatterBots}.\cite{aiconf}
\\

Up until today the \emph{Turing Test} has only been passed limited to certain domains and there is no chatbot yet that is able to simulate general human behavior indistinguishably from a real human being.
\\
It needs to be noted that, although creating an as human-like as possible system remains a popular challenge, not all applications of chatbots profit from this type of behavior.
Many systems are instead optimized to provide quick and efficient interactions and behave accordingly without attempting to hide their artificiality.

\section{Present Day}
\label{presentday}


With more than sixty years of history the concept of chatbots is not a recent discovery.
\\
People have been fascinated with the idea of being able to \emph{talk to computers} for a long time;
but past attempts have mostly been simple experiments or applications focused on the aspect of entertainment associated with machines inspired by \emph{science fiction}.
\\

Lately the technology industry and press are increasingly interested in the topic of conversational interfaces and chatbots.
\\
The reinforced interest can be explained by observing recent developments of technology and current market trends.
\\

Since Apple's release of \emph{Siri}~\cite{iphonelaunch} in 2011 customers have become more aware of the possibilities of conversational interfaces.
Even though the capabilities were limited at that time,
functionality improved quickly in the following years,
which can be attributed to the new competition Apple triggered in the market.
\\

At the same time artificial intelligence gained new traction due to the success of using \emph{artificial neural networks} for machine learning.
\\

The concept of artificial neural networks ``dates back to the 1950s, and many of the key algorithmic breakthroughs occurred in the 1980s and 1990s''~\cite{airevolution},
but only now they are successfully applied.
\\
This is mainly due to the increased computing power available today.
A second crucial condition is the necessity for a large disposable amount of data;
big Internet companies specialize on collecting data, originally intended to better target advertisement,
but now they can use their data to train artificial neural networks.
\\
The technology is named \emph{artificial neural network} because they are modeled after neural networks in the human brain;
instead of specifying rules of what a program should do, the machine learns from examples in similar ways to how humans learn.
Some tasks that are too complex to be solved with a rule-based program, can now be solved by collecting enough example data,
letting the machine figure out the solution instead.
\\
These techniques can also be applied in the field of natural language processing,
which is essential to understanding and generating text for conversational interfaces.
\\

With new technical possibilities more people see \emph{conversation as an interface} not only as an idea of science fiction movies,
but instead as something that could be possible in the real world.
\\

Also recent market trends make chatbots more compelling.
\\
``Computing is rapidly shifting to mobile devices''~\cite{mobileusage} and ``messaging apps have surpassed social networks in monthly active users''~\cite{convtrends}.
As a result of these developments users do not have space for complex interfaces on the small screens of their devices,
they need solutions light-weight in data consumption, and they can not use complicated keyboard shortcuts.
At the same time, users are already spending a majority of their time in instant messaging applications
and therefore, they are well accustomed to chatting as means of communication.
\\

Originating from the current state of technology the increasing interest in conversational interfaces leads to new platforms, products and applications for chatbots.

\section{Platforms}
\label{platforms}

Unless software is distributed on dedicated hardware, software products are designed to be executed by and accessed through other software.
The underlying software is the platform a product is created for.
\\
Products that target operating systems such as \emph{Microsoft Windows} or Apple's \emph{iOS} require users to install necessary executable files on their local system.
\\
Other software uses the \emph{web} as platform, whereby customers use a \emph{web browser} to access the software over the Internet,
while the software itself is executed on another computer referred to as \emph{server}.
\\

In the case of chatbots the target platform can be any medium that allows users to send messages to each other.
A chatbot can be seen as a counterpart to interact with in the same way users interact with other humans.
\\

There are numerous platforms available that fulfill these requirements.
\\

While messaging platforms provide means of communication, chatbots function similar to software accessible via a web browser;
a server executing the chatbot software is needed and the messaging platform communicates with the server in the same way a web browser does communicate with a server on the user's behalf.
\\

Because of the wide variety of available messaging platforms it is not possible to create an all-encompassing collection of available platforms in the context of this work;
the following is an overview over the currently most popular platforms, including their capabilities and their area of focus.
\\

To begin with, one of the most used online communication platforms is \emph{E-mail}.
\\
\emph{E-mail}, however, does not provide the earlier in \ref{defchatbot} on page \pageref{defchatbot} defined characteristics of chatbots to be able to communicate \emph{informally} and in \emph{real time};
which disqualifies \emph{E-mail} as a platform for chatbots, even though in practice many use cases of chatbots overlap with the ones that can be solved with the automation of \emph{E-mail}.
\\
Although users can choose to express themselves less formally and certain E-mail providers deliver E-mail in a very short time period,
this statement is based on the current general use case whereby these two attributes are not given.
Still, it is indeed possible for this characteristics to change in the future
and nothing fundamental about the E-mail protocols is preventing their usage for chatbots.
\\

A well-known communication technology, which is suited for chatbots, is \emph{Short Message Service}, short \emph{SMS}.
\\
\emph{SMS} is primarily used on mobile devices and users are identified by their phone numbers,
wherefore the communication has to happen through cell network providers.
The technology is limited in number of characters, often users are charged by amount of messages sent and communication is limited to text-only.
``End of year 2009 user level for SMS globally was 78\%, ie 3.6 Billion''\cite{mobilenumbers} people worldwide,
which means it remains one of the most popular communication channels;
and it therefore is an interesting option for applications requiring a low entry barrier.
\\

Since chatbots can communicate not only via text but also using voice, \emph{phone calls} are also a possible medium.
They are a common way of communication available to a large number of people.
\\
However, when relying solely on voice for communication without any visual feedback, the design of the user experience has to be thought out especially careful.
Furthermore to not only understand and generate natural language,
but to also parse and generate voice comes with further development costs.
\\

Apple's \emph{Siri} is another voice-based system available; but as of writing it is not accessible as platform for external services.
\\
Voice-based systems that can be targeted as platforms are Amazon's \emph{Alexa} and \emph{Google Assistant}.
Both systems are \emph{general assistants} helping the user with a variety of tasks
and in both cases tasks can be delegated to third parties.
\\

Currently popular target platforms for chatbots are \emph{messenger platforms}.
They are primarily text based, they mostly come without cost for the end-user and additional to text they often support multimedia formats such as pictures, audio, locations and stickers.
Some platforms also allow developers to display sliders, buttons and other graphical interface elements to the user, which can help guiding users instead of exclusively relying on natural language for communication.
\\
At this point it is not feasible to create a comprehensive list of available features for each platform, since the space is innovating constantly and many of the platforms add new features almost every single month.
\\

Facebook's \emph{WhatsApp} is with one billion active users in January 2017 one of the most popular messenger applications\cite{fbpopular}.
It, however, does currently not allow automated access to its platform and therefore using it as a chatbot platform is not a viable option.
\\

The second messenger application belonging to Facebook, called \emph{Facebook Messenger}, is equally popular with one billion active users in January 2017 as well\cite{fbpopular}.
Contrary to WhatsApp, Facebook Messenger provides a platform for developing chatbots.
Counting over 100.000 monthly active chatbots\cite{messenger}, it is an interesting platforms to develop for.
\\

Following in popularity\cite{appusage} are two messenger platforms from China, \emph{QQ Mobile} and \emph{WeChat}.
Both of them currently do not provide a specific chatbot platform,
but there have been successful attempts at creating chatbots for these platforms\cite{wechatbot}.
\\

Further popular messenger applications are the Japanese \emph{Line}, Microsoft's \emph{Skype}, \emph{Telegram} and the more business focused \emph{Slack}.
All of these applications provide dedicated platforms for the development of chatbots.
\\

The choice of platform primarily depends on the target market.
Different audiences prefer different platforms and a product might be better suited for certain environments.

\begin{figure}[H]
	\centering
	\includegraphics[width=0.7\textwidth]{images/similarweb-messenger-by-country.png}
	\caption{Most Popular Messaging App in Every Country\cite{similarweb}}
	\label{fig:similarweb}
\end{figure}

One important factor can be the geographical location of the target audience.
\\
As visible in figure \ref{fig:similarweb}, \emph{Facebook Messenger} and \emph{WhatsApp} are the global leading messengers,
and as previously mention the markets in China and Japan are dominated by \emph{WeChat} and \emphLine respectively,
but the data shows some lesser know trends; for example the \emph{Thai} market is also dominated by\emph{ Line}
and in \emph{Iran}, \emph{Telegram} is the most popular messenger application.


\subsection{Cross-platform Development}
\label{crossplatform}

As with the creation of other kinds of software, it is possible to release the same chatbot software for multiple target platforms,
whereby the interaction between software and platform has to conform to the technical details and protocols of each environment, the usage of platform-specific features has to be adapted individually and the user experience needs to be designed to fit each environment's expectations.
\\

There are existing frameworks that allow developers to develop a chatbot once and release it to multiple platforms at the same time without any adjustments to individual platforms.
\\
One such framework is \emph{API.ai} by \emph{Google}.
As of writing it supports 16 different integrations, including platforms such as \emph{Facebook Messenger}, \emph{Skye} and \emph{Slack}\cite{apiai}.
However, this platform is more than an adapter to different platforms.
It is complete solution to developing chatbots.
\emph{API.ai} comes with built-in support for natural language processing,
a chatbot is already able to have basic conversations out of the box,
and developers can train chatbots about new topics by just providing example conversations while the framework handles all of the language parsing.
\\
Detected keywords and intends can be forwarded to be handled with custom logic;
although for simple use cases this might not be necessary and a chatbot can be developed without writing a single line of code.
\\

Still, there are limits to platforms like \emph{API.ai}.
\\
First, intend parsing is, in the case if \emph{API.ai}, currently limited to a finite list of topics.
If a chatbot is handling topics from a domain unknown to \emph{API.ai}, this solution is not sufficient anymore.
\\
Further, developers have no control over the applied machine learning and natural language processing algorithms.
There are no possibilities for customization, if the parsing results or the generated responses do not match the requirements.
\\
Additionally, while \emph{API.ai} currently supports 15 different languages, a chatbot is limited to the available languages.
If a new language needs to be supported, a lot of work might be necessary switching to a custom solution.
\\
Another issue to keep in mind is, that, while \emph{API.ai} has support for many platform-specific features such as custom formats for message content and \emph{quick reply} buttons,
there are unique features of single platforms that are not supported
and since the space is evolving at such rapid pace future extentions might not be available either.
\\
As last point it should be mentioned that, even though \emph{API.ai} is at the moment free to use for everyone,
they will very likely search for a sustainable business model in the near future.
This business model could be simply charging for the service or collecting data which Google can use elsewhere, binding customers and gaining market share.
\\
Regardless which monetization strategy is chosen, as a developer one has to be aware that such a service is a non-controllable, external dependency.

\input{content/09-products.tex}
% NOTE: skip these 3 chapters for now, maybe I don't even need them
% \input{content/10-innovative-ideas.tex}
% \input{content/11-companies.tex}
% \input{content/12-usage.tex}
\section{Promises}


As explained previously in \ref{presentday}, the interest in chatbots and conversational interfaces increased in recent time, because of the advancements made in the field of artificial intelligence and the popularity of messaging platforms, mobile devices, and personal assistants like Siri.
\\
The conditions are right to think about using the newly available possibilities, however, the question remaining is why we would want chatbots. What can chatbots achieve that existing solutions are not good at, both, from a user's point of view and looking at the interests companies have.
\\

From a user's point of view chatbots can be seen as a new interface to interact with computers.
Existing interfaces are not intuitive to humans. Using technology is something humans need to learn first.
With every new application one uses and every new website one visits there is a new interfaces to adapt to.
``Adjusting to a machine does not come naturally to us. With every app you need to learn how to use it. ... Conversations come naturally to us''\cite{techinasia}.
Conversation is a way of communicating humans already know how to use because they use it to interact with other humans.
If it is possible to use this communication technology to interact with machines, it should be really intuitive for humans to use it.
``The vision for a chatbot: get machines to respond to questions like a human being''\cite{techinasia}.
\\

Further, there is a trend in consumer behavior of ``outsourcing their "chores", such as driving, shopping, cleaning, food delivery, errands''\cite{chatbotbook} to companies that offer these services.
Service companies are not a new occurrence, however, in the past it took more effort to coordinate the usage of such services.
By using technology to automate many steps of the coordination process, not only the cost can be lowered, but also the friction for customers using a service is reduced significantly.
Managing and coordinating the usage of such services is a task conversational interfaces are particularly suited for because it is a scenario that profits especially from the simplicity and low friction that characterize conversational interfaces.
\\

Users can also profit from technical advantages of chatbots over to native applications and websites.
\\
Native applications need to be downloaded first which includes all resources and not just the ones we require at this moment.
Websites are a little slimmer and one only needs to download the resources required to load the current page.
But one page still contains not only content, but also layout information, styling, decorative images and in most cases also Javascript to run some additional logic in the web browser.
\\
The only thing a chatbots needs to download over the network is the content. Everything else is provided my the platform it is embedded in.
Compared to these existing solutions, ``a chatbot uses very low bandwidth''\cite{techinasia}, which can be an important advantage not only for the perceived responsiveness but especially in places where slow network connections are still common.
\\

Providing customers a more intuitive and more direct way of interacting with a company's product is already a compelling reason for a company to be interested in the new platforms, but there are additional benefits companies can draw from conversational interfaces, which are not perceivable for the users.
\\

First, ``the cost of developing a chatbot is one-third of what is required in developing a mobile app''\cite{techinasia}. This might not be the case for every single product but in general, creating a chatbot is less work than creating a mobile application, because there is no custom design required nor  is it necessary to write code for the logic controlling the user interface.
\\

Next, ``Chat apps also have higher retention and usage rates than most mobile apps''\cite{businessinsider}. Since chatbots are part of a chat application they can take advantage of being where the attention of mobile phone users already is. A chatbot can therefore potentially gain more user engagement than a competing website or mobile application.

Another aspect of the flexible nature of using natural language as an interface is that ``chatbots are able to gain invaluable data and insights on user behavior''\cite{drum}, because firstly, users have the freedom to send any kind of information and feedback and secondly, being in the context of a conversation people tend to be more talkative than they would be in a more formal environment.
Especially for companies such as media outlets or retailers being able to further profile users can be a useful assistance in tailoring personal experiences for users and targeting them with individual offers.
\\
Additional context and user data is also available on the platform itself; when interacting with a user via a chatbot on the Facebook Messenger platform all public information of the user's Facebook profile is also available for the company to user for further personalization.
\\

Lastly, the most fundamental reasons for a company to be interested in chatbots as a platform are the before mentioned popularity and ubiquity of messenger applications.
``The question brands and publishers now face is how to engage with these private social network users''\cite{drum}.
When the attention of users is shifting away from not only non-digital media but also away from other mobile applications, including traditional social networks, companies need to find a way to reach users at the place they spend most of their time at.


% APIs ``can be accessed from computers to complete real world tasks''\cite{chatbotbook}
% Already covered in previous chapter:
%``After nearly a decade of explosive growth, mobile apps have largely stopped growing''\cite{chatbotbook}
%``Social and messaging apps emerge as big winners''\cite{chatbotbook}
% ``Artificial Intelligence has gotten a lot better''\cite{chatbotbook}
% Graphic titled ``Messaging Apps Have Surpassed Social Networks''\cite{businessinsider}

\chapter{Development}


To not only understand possible applications of chatbots but also understand what the practical development of a chatbot looks like this chapter guides through the development of an example chatbot.
\\

To start with, an appropriate application needs to be chosen and specified in its requirements.
Before starting with the implementation possible usage scenarios need to be defined and matching user stories will be created.
\\
When all requirements are set, the appropriate platforms, tooling and solutions can be selected.
After all preparations are done the technical implementation will take please.
It is followed by any analysis of complications and a comparison to other possible solutions.


% \section{Choosing a Practical Example}

Before thinking about implementation details a suitable application for a chatbot needs to be selected.
\\
As illustrated earlier, chatbots can be used to cater a wide variety of applications.

Since this example application should be a suitable demonstration of the different aspects of developing chatbots,
it should not be too simplistic in scope.
An appropriate example covers more than one of the product categories described in section \ref{classification},
while being, at the same time, not too technical challenging in a problem domain which is not a specific to chatbot development.
\\
A service, that accepts image files and returns the same picture with a filter applied, might be an interesting and entertaining use case for a chatbot, however, it would not be an adequate example to give a general introduction to chatbot development since, even though it would a technical interesting task to solve regarding image processing, it would not illustrated much technical details in the domain of chatbot development.
\\

The here selected example application is a system for individual language studying.
\\
While this idea arises from a personal need for such a system and an observed shortage in the functionalities existing solutions provide,
this application also fulfills the stated requirements of a suitable example application.
The language studying chatbot fits into multiple product categories from section \ref{classification}; it is not only an optimization in an existing problem domain but also makes use of proactive features, group scenarios and eventually entertaining aspects.
\\
Although there is a necessity to understand parts of the problem domain of language studying,
the required domain-specific knowledge is minimal and the main communication medium remains text-based.
\\

The idea is to build a system independent from existing learning resources, that users can use to study their own vocabulary.
Since there is no need to focus on content for specific languages, the main focus remains the implementation of the chatbot features.
\\
In short, a user is able to input new vocabulary; then the system tests the user's knowledge in appropriate learning intervals.

% \section{Existing Solutions} \label{existing}

Before starting a new projects it is helpful to research for existing solutions that might solve the same problem.
\\
There is already a variety of existing software applications for language studying, which have very different use cases and solve different problems.

One significant separation is between software that includes content and software the user can customize to study personal content.
\\

The first segment is the most prominent. This software is intended to enable people to self-study language and, at least partially, replace physical language courses.
\\
A main reason for the prominence of this segment can be attributed to the ability to sell content.
Professionally curating a language course curriculum requires teaching expertise and is a lot of effort, and therefore content remains expensive, not only in software but also in the form of physical textbooks.
\\
One popular example from this segment is Duolingo. ``Duolingo has courses in a handful of languages.... The courses are structured in a way like games as well—you earn skill points as you complete lessons''\cite{lifehacker}.
\\
Interestingly, Duolingo recently released a chatbot\cite{topbots2} as part of their iPhone application which enables the user to have a written conversation about certain topics with a chatbot and to thereby learn the appropriate phrases for the given scenario. Although the topic and possible phrases are restricted in each scenario, this is a first example of how chatbots can be used for language studying.

% TODO: think about metioning Mondly and Eggbun here
% https://blog.mondlylanguages.com/2016/08/25/mondly-chatbot-press-release/
% http://www.tofugu.com/japanese/japanese-learning-resources-april-2017/#eggbun


The second segment consists of software which doesn't not provide users a guideline what to study but is instead intended to support users studying their own content.
\\
Most of the software that can be found in this segment is a variation of the attempt to bring traditional flashcards to digital media.
\\
One of the most established software in this segment is Anki\cite{lifehacker}, which exists for more than ten years already and provides a flexible, but also rather complex, interface to create a various kinds of studying material.
\\
Another more recent competitor is Memrise, where users get a more intuitive interface, which also includes several gamification\footnote{The Oxford Dictionaries describe gamification as ``the application of typical elements of game playing (e.g. point scoring, competition with others, rules of play) to other areas of activity''\cite{oxfordgamification}} features to make the studying process more appealing.
\\
Both mentioned products are not restricted to a field of study and users are able to add their own content. Furthermore both products also offer mechanisms for users to share content with each other, which allows users to reuse what other users created instead.
\\

For the here planned chatbot example the second segment is more fitting, since there are no resources in the current context to curate professional content.
The following implementation is an attempt at creating a software product with a conversational text interface that is in its use cases similar to products like Anki and Duolingo while using the unique features chatbots as a medium provide compared to the here mentioned products.

% \section{Definition of Use Cases}

Building on the analysis of existing solutions in \ref{existing}, features for the chatbot need to be specified.
\\
An effective method for gathering crucial features is by finding potential users
and creating usage scenarios for their individual needs.
\\

To apply this method, the fundamental problem the application is solving needs to be defined.
\\
The issue this chatbot is trying to help with is the study of individual vocabulary.
The goal is not to provide studying material in a way a language course or a textbook does,
but instead to complement these resources with a tool to study new vocabulary and phrases the learner picked up
while studying or in a different situation in every day life.
\\


\subsection{User Stories}

The following are two individuals that might possibly use the chatbot and both of them profit from it in different ways.
\\

Clara is a 22 years old American.
She moved to New York City to go to University.
Currently she's in the last year for her Bachelor degree in economic.
In University she signed up for an evening class in Mandarin.
She uses Facebook Messenger every day to talk to her friends and when a friend sent her a link
she found the chatbot.
For her the most difficult part of the studies is to write hànzì, the Chinese characters.
Now Clara uses the chatbot to write down vocabulary in hànzì during her class,
and at home she revises the new characters by going through them using the chatbot and writing the characters
down on paper.
\\

Pierre is 29 years and born in Bordaux in France.
He studied computer science.
A year ago he moved to Berlin where he found a job at a startup.
At work everyone all communication is happening in English since the team consists of people from all around the globe.
Because Pierre is not a native English speaker, he picks up new words at work almost every single day.
Since moving to Berlin Pierre also made a few German friends and he tries to pick up new words they teach him
and he also tries to remember things he sees in the supermarket.
He found the chatbot on a news website for technology products,
and since then whenever Pierre learns a new word he gets his phone from his pocket and adds the word to the chatbot.
Since the chatbot has no restrictions on what to learn, Pierre uses it to save both, German and English, vocabulary in one place.
Pierre's daily commute from and to work takes him 40 minutes each.
Now he takes advantage of this time by taking out his phone and reviewing new vocabulary he picked up the previews days.
\\


\subsection{Features}

The above defined user stories can be used to extract all necessary features.
\\

First, a user needs to be able to add new vocabulary.
\\
There should not be any restrictions on what to add
and vocabulary should not be limited to single words because in many cases it is more useful
to add whole phrases instead.
\\
Each vocabulary consists of the phrase the user is trying to remember
and an explanation to help the user understand what the phrase means.
\\

Next, users need a way to revise vocabulary.
\\
There should be two possible modes for revising;
one version where users can decide on their own when to go to the next phrase
and of they remembered the phrase correctly,
and a second way whereby users type out the phrase right in the messenger application.

\\
Last, it is necessary to have a means of deciding what to study next.
\\
A user should not be required to think about what to review or even when it is the right to review vocabulary.
The chatbot needs a system to decide the review time for each vocabulary,
and ideally the user is notified when vocabulary is ready to be reviewed by sending a message to the user.
\\



These three main features can be seen as a sufficient minimal viable product, MVP, for this chatbot.
\\
For demonstration purposes it is desired to keep the product as simple as possible.
\\
The knowledge that can be taken from making decisions about the implementation and walking through the process of creating the chatbot,
is mostly independent from this particular product and can be applied to the development of similar chatbot products.


% \input{content/18-platform-evaluation.tex}
% \section{Setup Overview}

-	List technologies used (Languages, Databases, Libraries, Server setup, ...)
-	Give overview about architecture
-	Including Graphic

% \input{content/20-feature-implementation.tex}
% \input{content/21-technical-details.tex}
% \input{content/22-obstacles-and-limitations.tex}
% \input{content/23-conclusions.tex}
% \input{content/24-outlook-and-possibilities.tex}


\printbibliography[heading=bibintoc]


\chapter*{Erklärung}

Hiermit versichere ich, dass ich die vorliegende Arbeit selbstständig verfasst und keine anderen als die angegebenen Quellen und Hilfsmittel benutzt habe, dass alle Stellen der Arbeit, die wörtlich oder sinngemäß aus anderen Quellen übernommen wurden, als solche kenntlich gemacht und dass die Arbeit in gleicher oder ähnlicher Form noch keiner Prüfungsbehörde vorgelegt wurde.

\vspace{3cm}
Ort, Datum \hspace{5cm} Unterschrift\\

\end{document}
