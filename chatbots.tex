% \documentclass[12pt, openright]{report}
\documentclass[12pt, headinclude, parskip=half+, numbers=noenddot, openright]{scrreprt}
% PDF-Kompression
\pdfminorversion=5
\pdfobjcompresslevel=1
% Allgemeines
\usepackage[automark]{scrpage2} % Kopf- und Fußzeilen
% \usepackage{amsmath,marvosym} % Mathesachen
\usepackage[T1]{fontenc} % Ligaturen, richtige Umlaute im PDF
\usepackage[utf8]{inputenc}% UTF8-Kodierung für Umlaute usw
% Schriften
\usepackage{setspace} % Zeilenabstand
\usepackage{titlesec}
\onehalfspacing % 1,5 Zeilen
% Schriften-Größen
\setkomafont{chapter}{\Large\rmfamily} % Überschrift der Ebene
\setkomafont{section}{\large\rmfamily}
\setkomafont{subsection}{\large\rmfamily}
\setkomafont{subsubsection}{\normalsize\rmfamily}
\setkomafont{chapterentry}{\large\rmfamily} % Überschrift der Ebene in Inhaltsverzeichnis
\setkomafont{descriptionlabel}{\bfseries\rmfamily} % für description Umgebungen
\setkomafont{captionlabel}{\small\bfseries}
\setkomafont{caption}{\small}
% Language: English
\usepackage[english]{babel}
% PDF
\usepackage[final]{microtype} % mikrotypographische Optimierungen
\usepackage{float}
\usepackage{url} % ermögliche Links (URLs)
\usepackage{pdflscape} % einzelne Seiten drehen können
\usepackage{hyperref}
% Tabellen
\usepackage{multirow} % Tabellen-Zellen über mehrere Zeilen
\usepackage{multicol} % mehre Spalten auf eine Seite
\usepackage{tabularx} % Für Tabellen mit vorgegeben Größen
\usepackage{longtable} % Tabellen über mehrere Seiten
\usepackage{array}
%  Bibliographie
% Sort by order of appearance in document
\usepackage[sorting=none]{biblatex}
% Bilder
\usepackage{graphicx} % Bilder
\usepackage{color} % Farben
\graphicspath{{images/}}
\DeclareGraphicsExtensions{.pdf,.png,.jpg} % bevorzuge pdf-Dateien
\usepackage{subcaption}  % mehrere Abbildungen nebeneinander/übereinander
\usepackage[all]{hypcap} % Beim Klicken auf Links zum Bild und nicht zu Caption gehen
% Bildunterschrift
\setcapindent{0em} % kein Einrücken der Caption von Figures und Tabellen
\setcapwidth{0.9\textwidth} % Breite der Caption nur 90% der Textbreite, damit sie sich vom restlichen Text abhebt
\setlength{\parskip}{0pt}
\setlength{\abovecaptionskip}{0.2cm} % Abstand der zwischen Bild- und Bildunterschrift
% Quellcode
% für Formatierung in Quelltexten, hier im Anhang
\usepackage{listings}
\definecolor{grau}{gray}{0.25}
\lstset{
	extendedchars=true,
	basicstyle=\normalsize\ttfamily,
	tabsize=2,
	keywordstyle=\textbf,
	commentstyle=\color{grau},
	stringstyle=\textit,
	numbers=left,
	numberstyle=\tiny,
	% für schönen Zeilenumbruch
	breakautoindent  = true,
	breakindent      = 2em,
	breaklines       = true,
	postbreak        = ,
	prebreak         = \raisebox{-.8ex}[0ex][0ex]{\Righttorque},
}
% linksbündige Fußboten
\deffootnote{1.5em}{1em}{\makebox[1.5em][l]{\thefootnotemark}}

\typearea{14} % typearea berechnet einen sinnvollen Satzspiegel (das heißt die Seitenränder) siehe auch http://www.ctan.org/pkg/typearea. Diese Berechnung befindet sich am Schluss, damit die Einstellungen oben berücksichtigt werden

\usepackage{scrhack} % Vermeidung einer Warnung



\addbibresource{references.bib}



\begin{document}



\pagenumbering{Roman}
\pagestyle{empty} % remove header and footer for first page

% title page
\clearscrheadings\clearscrplain
\begin{center}

HTW Berlin\\
University of Applied Sciences

\vspace{2cm}

Bachelor's Thesis

\vspace{1cm}

\begin{Large}
  Chatbots: Development and Applications
  \vspace{3cm}
\end{Large}

\begin{tabular}{rl}
{\bfseries Author} & Jorin Vogel\\
{\bfseries First Examiner} & Prof. Dr. Gefei Zhang\\
{\bfseries Second Examiner} & Prof. Dr. Carsten Busch\\
\end{tabular}

\vspace{3cm}

International Media and Computing\\
Faculty IV

\vspace{6cm}

July 29, 2017

\end{center}

\clearpage

\pagestyle{useheadings} % enable header and footer for other pages



\tableofcontents
\listoffigures
\listoftables

\pagenumbering{arabic}


% \chapter{Introduction}

\pagestyle{useheadings} % enable header and footer for other pages
\pagenumbering{arabic}


% What's the problem to be solved?
This work gives a general introduction to chatbots
by explaining what they are, what they can be used for and how to develop them.
\\
No previous domain-specific knowledge is required.
\\

% Why is it a problem?
Lately as of writing topics around chatbots have received increasing attention from media and also numerous investments from different actors in the industry.
\\
At the same time not many potential users know about the existence of chatbots or about areas in which chatbots could be helpful assistance.
The topic is equally unknown to developers.
\\
While the term \emph{chatbot} is commonly used in media, the meaning mostly remains ambiguous.
\\
There is a need for further explanation of what chatbots are and further analysis to identify well suited applications for chatbots.
\\
Additionally to spreading knowledge about the potentials of chatbots and their use cases,
more developers should be enabled to create new, innovative chatbots.
\\

% How do I intent to solve it?
The lack of knowledge can be solved by providing answers to the questions of what chatbots are, what benefits they bring and how to create them.
\\
An appropriate definition of chatbots can be given by analyzing the fundamental meaning of the term \emph{chatbot} and by exploring past and current applications.
\\
Use cases of chatbots can be identified by exploring existing applications. Additionally, market trends and attributes of media and technologies can be analyzed to find new potential scenarios for the usage of chatbots.
\\
Development is best explained by creating a real chatbot and by using it to present the general principles of the development process.
\\

% How will this help the world?
Explaining what chatbots are, demystifying what to use them for and presenting how to create them,
will help more people to be able to use and create chatbots, and thereby, accelerate the development of the chatbot ecosystem.
\\
Innovation in technology and the creation of new solutions can help automating and simplifying more tasks,
which gives people the opportunity to focus on more interesting issues and accomplish more things.
\\
Chatbots have the potential to simplify and automate many existing tasks and thereby accelerate the overall technological progress.
\\

% How is this work structured?
The structure of this work follows the three main questions.
\\
To begin with, terminology is defined and applications are explored to form a definition and understanding of what chatbots are.
\\
Afterwards use cases of chatbots are identified not only through the collection of existing examples, but also through the exploration of future potentials by analyzing attributes of the relevant technologies.
\\
The second half of the work is a case study for the development of a chatbot.
The presented example guides through the process of designing user interactions for a chatbot, and additionally explains architectural decisions and technological choices, which provide a basis for other developers to build on when creating new chatbots in the future.

\input{chapters/02-definition-chatbot.tex}
About Conversational Interfaces


Having defined the concept and character of chatbots, one apparent attribute of the technology under examination is that its domain of application involves interaction with one or more users.

Any form of interaction requires an interface as a way to interact.
An **interface** is generally described as a *"point where two systems, subjects, organizations, etc. meet and interact"* and in the area of computing it can be further defined as a *"device or program enabling a user to communicate with a computer"*. [^1]

This definition leaves a broad array of possible manifestations.
However, the range of suitable means of communication can be further narrowed by including another aspect of the previous definition, namely that interaction happens in the form of conversation.

Many communication mechanisms can be excluded by focusing on this characteristic of the interaction.
<br>
The term *conversation* strongly suggests the usage of natural, human language as a means of interaction while discouraging the idea of using an interface consisting purely of static or graphical elements.
<br>
Picturing a *conversational* interface one might expect a flow of interactions consisting of elements from all involved parties similar to the interfaces humans use to interact with each other.
<br>
It should also be noted that conversations are not limited to written language. Spoken language is also a possible interface for communication.
<br>

Focusing on written, text-based communication, it becomes apparent that not all text-based interfaces fit the characteristics of a *conversational interface*.
<br>
Classical command line interfaces often used to interact with computers are one example of text-based interfaces which don't have the attributes of conversational interfaces.
As implied in the naming, these interfaces use *commands* for interaction.
A command is an *"authoritative order"* [^2] which contrasts with a conversation.
<br>
The term *conversation* has a connotation of complex, non-linear communication where each involved party understands the underlaying ideas communicated opposed to merely receiving the characters the words consist of.

This deeper understanding of the intends a user has and the ability to adjust the interaction with a conversation consisting of customized parts sets conversational interfaces, and thus chatbots, apart from other interfaces.



[^1]: https://en.oxforddictionaries.com/definition/interface
[^2]: https://en.oxforddictionaries.com/definition/command

\input{chapters/04-applications.tex}
\input{chapters/05-history.tex}
\input{chapters/06-present-day.tex}
\input{chapters/07-platforms.tex}
\input{chapters/08-products.tex}
% \input{chapters/09-innovative-ideas.tex}
% \input{chapters/10-companies.tex}
% \input{chapters/11-usage.tex}
\input{chapters/12-potentials-and-promises.tex}
% \input{chapters/13-readme.tex}
% \section{Choosing a Practical Example}


A suitable application for a chatbot needs to be selected before thinking about implementation details.
\\
As illustrated earlier, chatbots can be used to cater a wide variety of applications.
\\

Since this application should be a demonstration of the different aspects of developing chatbots,
it should not be too simplistic in scope.
An appropriate example covers more than one of the product categories described in section \ref{classification} on page \pageref{classification},
while being, at the same time, not too technical challenging in a specific  problem domain outside of chatbot development.
\\
A service, that accepts image files and returns the same picture with a filter applied,
might be an interesting and entertaining use case for a chatbot,
however, it would not be an adequate example to give a general introduction to chatbot development since,
even though it would be a technical interesting task to solve regarding image processing,
it would not illustrate much technical details in the domain of chatbot development.
\\

The here selected example application is a system for \textbf{individual language studying}.
\\
While this idea arises from a personal need for such a system and an observed shortage in the functionalities existing solutions provide,
this application also fulfills the stated requirements of a suitable example application.
\\

The language studying chatbot covers multiple of the categories defined in section \ref{classification}.
\\
It is mainly an \emph{optimization chatbot} improving the task of language studying,
but the chatbot can also be classified as a \emph{proactive chatbot} because it can use proactive features to notify the user when it is time for studying.
\\

Although there is a necessity to understand parts of the problem domain of language studying,
the required domain-specific knowledge is minimal.
\\

The idea is to build a system independent from existing learning resources that users can use to study their own vocabulary.
Since there is no need to focus on content for specific languages, the main focus remains implementing the chatbot features,
which are primarily, that a user is able to input new vocabulary, and then the system tests the user's knowledge in appropriate studying intervals.

% \section{Existing Solutions}
\label{existing}


When starting a new project, it is helpful to research existing solutions which solve similar problems.
There are already a variety of existing software applications for language studying, which have various use cases and solve different problems.
\\
One significant separation is between software that includes content and software users can utilize and customize to study personal content.
\\

The first segment is the most prominent.
This software is intended to enable people to self-study language and, at least partially, replace physical language courses.
A main reason for the prominence of this segment is the ability to sell content.
Professionally curating the curriculum for a language course requires teaching expertise and takes a lot of effort.
Because of this, good content for language studying remains expensive, not only in software but also in the form of physical textbooks.
One popular example from this segment is \emph{Duolingo}. ``Duolingo has courses in a handful of languages.... The courses are structured in a way like games as well - you earn skill points as you complete lessons''~\cite{lifehacker}.
\\
Interestingly, \emph{Duolingo} recently released a chatbot~\cite{topbots2} as part of their \emph{iPhone} application which enables the user to have a text-based conversation about certain topics with a chatbot to learn the appropriate phrases for the given scenario.
Although the topics and possible phrases are restricted in each scenario, this is a first example of how chatbots can be used for language studying.
\\

The second segment consists of software which does not provide users a guideline for what to study, but is instead intended to support users studying their own vocabulary.
Most of software belonging to this segment are attempts at bringing traditional flashcards to digital media.
\\
One of the most established applications in the second segment is \emph{Anki}~\cite{lifehacker},
which exists for more than ten years already and provides a flexible,
but also rather complex, interface to create various kinds of studying material.
Another more recent competitor is \emph{Memrise}, which gives users a more intuitive interface, that also includes several gamification\footnote{The Oxford Dictionaries describe gamification as ``the application of typical elements of game playing (e.g. point scoring, competition with others, rules of play) to other areas of activity''~\cite{oxfordgamification}} features to make the studying process more appealing.
Both mentioned products are not restricted to a field of study and users are able to add their own content.
Furthermore both products also offer mechanisms for users to share content with others, which allows users to reuse what other users created.
\\

For the in this work planned chatbot example the second segment is more fitting, since there are no resources in the current context to curate professional content.
\\

The following implementation is an attempt at creating a software product with a conversational text interface that is in its use cases similar to products like \emph{Anki} and \emph{Duolingo} while making use of the unique features the medium \emph{chatbot} provides.

% \section{Use Cases and Requirements}


Building on the analysis of existing solutions in \ref{existing} on page \pageref{existing}, features for the chatbot need to be specified.
\\
An effective method for gathering crucial features is by finding potential users
and creating usage scenarios for their individual needs.
\\

To apply this method, the fundamental problem the application is solving needs to be defined.
\\

The issue this chatbot is trying to help with is the study of individual vocabulary.
The goal is not to provide studying material in a way a language course or a textbook does,
but instead to complement these resources with a tool to study new vocabulary and phrases the student picked up
while studying or in a different situation in every day life.
\\


\subsection{User Stories}

The following are two hypothetic individuals that might use the chatbot and both profit from it in different ways.
\\

Clara is a 22 years old American.
She moved to New York City to go to university.
Currently she is in the last year for her bachelor degree in economic.
In university she signed up for an evening class in Mandarin.
She uses \emph{Facebook Messenger} every day to talk to her friends and she discovered the chatbot when a friend sent her a link.
For her the most difficult part of the studies is to write hànzì, the Chinese characters.
Now Clara uses the chatbot to write down vocabulary in hànzì during her class,
and at home she revises the new characters by going through them using the chatbot and writing the characters
down on paper.
\\

Pierre is 29 years and born in Bordeaux in France.
He studied computer science.
A year ago he moved to Berlin where he found a job in a startup.
At work all communication is happening in English since the team consists of people from all around the globe.
Because Pierre is not a native English speaker, he picks up new words at work almost every single day.
Since moving to Berlin Pierre also made a few German friends and he tries to pick up new words they teach him.
He found the chatbot on a news website for technology products,
and since then whenever Pierre learns a new word he grabs his phone from his pocket and adds the word to the chatbot.
Since the chatbot has no restrictions on what to learn, Pierre uses it to save both, German and English, vocabulary in one place.
Pierre's daily commute from and to work takes 40 minutes twice a day.
Now he uses his commute time to take out his phone and review new vocabulary he picked up the previews days.
\\


\subsection{Functional Requirements}
\label{funcreq}

All necessary functional requirements can be extracted from the above defined user stories.
\\

First, a user needs to be able to add new vocabulary.
\\
There should not be any restrictions on what to add
and vocabulary should not be limited to single words because in many cases users prefer
to add whole phrases instead.
\\
Each vocabulary consists of the phrase the user is trying to remember
and an explanation to help to understand the meaning of the phrase.
\\

Next, users need a way to revise vocabulary.
\\
There should be two possible modes for revising;
one version where users can decide on their own when to go to the next phrase
and if they remembered the phrase correctly,
and a second way whereby users type out the phrase right in the messenger application.
\\
In each case the system should keep track of whether the user knew the correct solution or not.
\\

Lastly, it is necessary to have a means of deciding what to study next.
\\
A user should not be required to think about what to review or about the right time to review vocabulary.
The chatbot needs a system to decide the review time for each vocabulary,
and ideally the user is notified when vocabulary is ready to be reviewed by sending a message to the user.
\\

These three main features can be seen as a sufficient \emph{minimal viable product}, MVP, for this chatbot.
\\

For demonstration purposes it is desired to keep the product as simple as possible.
\\
The knowledge that can be taken from making decisions about the implementation and walking through the process of creating the chatbot,
is mostly independent from this particular product and can be applied to the development of other chatbot products.

% \input{chapters/17-platform-evaluation.tex}
% \input{chapters/18-setup-overview.tex}
% \section{Feature Implementation}

-	Presentation of the product
-	Show some screenshots for usage of features
-	Explain how to use features
-	Explain how each feature is implemented


Could add subsection about testing here.
How to test a bot seems kind of tricky.

% \input{chapters/20-technical-details.tex}
% \input{chapters/21-obstacles-and-limitations.tex}
% \input{chapters/22-conclusions.tex}
% \input{chapters/23-outlook-and-possibilities.tex}
% \input{chapters/24-references.tex}


\printbibliography[heading=bibintoc]


\chapter*{Erklärung}

Hiermit versichere ich, dass ich die vorliegende Arbeit selbstständig verfasst und keine anderen als die angegebenen Quellen und Hilfsmittel benutzt habe, dass alle Stellen der Arbeit, die wörtlich oder sinngemäß aus anderen Quellen übernommen wurden, als solche kenntlich gemacht und dass die Arbeit in gleicher oder ähnlicher Form noch keiner Prüfungsbehörde vorgelegt wurde.

\vspace{3cm}
Ort, Datum \hspace{5cm} Unterschrift\\

\end{document}
