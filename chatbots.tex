% \documentclass[12pt, openright]{report}
\documentclass[12pt, headinclude, parskip=half+, numbers=noenddot, openright]{scrreprt}
% PDF-Kompression
\pdfminorversion=5
\pdfobjcompresslevel=1
% Allgemeines
\usepackage[automark]{scrpage2} % Kopf- und Fußzeilen
% \usepackage{amsmath,marvosym} % Mathesachen
\usepackage[T1]{fontenc} % Ligaturen, richtige Umlaute im PDF
\usepackage[utf8]{inputenc}% UTF8-Kodierung für Umlaute usw
% Schriften
\usepackage{setspace} % Zeilenabstand
\usepackage{titlesec}
\onehalfspacing % 1,5 Zeilen
% Schriften-Größen
\setkomafont{chapter}{\Large\rmfamily} % Überschrift der Ebene
\setkomafont{section}{\large\rmfamily}
\setkomafont{subsection}{\large\rmfamily}
\setkomafont{subsubsection}{\normalsize\rmfamily}
\setkomafont{chapterentry}{\large\rmfamily} % Überschrift der Ebene in Inhaltsverzeichnis
\setkomafont{descriptionlabel}{\bfseries\rmfamily} % für description Umgebungen
\setkomafont{captionlabel}{\small\bfseries}
\setkomafont{caption}{\small}
% Language: English
\usepackage[english]{babel}
% PDF
\usepackage[final]{microtype} % mikrotypographische Optimierungen
\usepackage{float}
\usepackage{url} % ermögliche Links (URLs)
\usepackage{pdflscape} % einzelne Seiten drehen können
\usepackage{hyperref}
% Tabellen
\usepackage{multirow} % Tabellen-Zellen über mehrere Zeilen
\usepackage{multicol} % mehre Spalten auf eine Seite
\usepackage{tabularx} % Für Tabellen mit vorgegeben Größen
\usepackage{longtable} % Tabellen über mehrere Seiten
\usepackage{array}
%  Bibliographie
% Sort by order of appearance in document
\usepackage[sorting=none]{biblatex}
% Bilder
\usepackage{graphicx} % Bilder
\usepackage{color} % Farben
\graphicspath{{images/}}
\DeclareGraphicsExtensions{.pdf,.png,.jpg} % bevorzuge pdf-Dateien
\usepackage{subcaption}  % mehrere Abbildungen nebeneinander/übereinander
\usepackage[all]{hypcap} % Beim Klicken auf Links zum Bild und nicht zu Caption gehen
% Bildunterschrift
\setcapindent{0em} % kein Einrücken der Caption von Figures und Tabellen
\setcapwidth{0.9\textwidth} % Breite der Caption nur 90% der Textbreite, damit sie sich vom restlichen Text abhebt
\setlength{\parskip}{0pt}
\setlength{\abovecaptionskip}{0.2cm} % Abstand der zwischen Bild- und Bildunterschrift
% Quellcode
% für Formatierung in Quelltexten, hier im Anhang
\usepackage{listings}
\definecolor{grau}{gray}{0.25}
\lstset{
	extendedchars=true,
	basicstyle=\normalsize\ttfamily,
	tabsize=2,
	keywordstyle=\textbf,
	commentstyle=\color{grau},
	stringstyle=\textit,
	numbers=left,
	numberstyle=\tiny,
	% für schönen Zeilenumbruch
	breakautoindent  = true,
	breakindent      = 2em,
	breaklines       = true,
	postbreak        = ,
	prebreak         = \raisebox{-.8ex}[0ex][0ex]{\Righttorque},
}
% linksbündige Fußboten
\deffootnote{1.5em}{1em}{\makebox[1.5em][l]{\thefootnotemark}}

\typearea{14} % typearea berechnet einen sinnvollen Satzspiegel (das heißt die Seitenränder) siehe auch http://www.ctan.org/pkg/typearea. Diese Berechnung befindet sich am Schluss, damit die Einstellungen oben berücksichtigt werden

\usepackage{scrhack} % Vermeidung einer Warnung



\addbibresource{references.bib}



\begin{document}



\pagenumbering{Roman}
\pagestyle{empty} % remove header and footer for first page

% title page
\clearscrheadings\clearscrplain
\begin{center}

HTW Berlin\\
University of Applied Sciences

\vspace{2cm}

Bachelor's Thesis

\vspace{1cm}

\begin{Large}
  Chatbots: Development and Applications
  \vspace{3cm}
\end{Large}

\begin{tabular}{rl}
{\bfseries Author} & Jorin Vogel\\
{\bfseries First Examiner} & Prof. Dr. Gefei Zhang\\
{\bfseries Second Examiner} & Prof. Dr. Carsten Busch\\
\end{tabular}

\vspace{3cm}

International Media and Computing\\
Faculty IV

\vspace{6cm}

July 29, 2017

\end{center}

\clearpage

\pagestyle{useheadings} % enable header and footer for other pages



\tableofcontents
\listoffigures
\listoftables

\pagenumbering{arabic}


% \chapter{Introduction}

Notes

-	What's the problem to be solved?
-	Why is it a problem?
-	How do I intent to solve it?
-	How will this help the world?
-	How is this work structured?


There have been many headlines about chatbots in recent time; yet still few people use them in their daily lives. That makes it uncertain in which businesses chatbots and their conversational user interfaces may be an improvement over the current, graphical interfaces.

In this thesis I would like to develop a chatbot specifically with the goal in mind to create a product people would actually use. Additionally, building on the experience of implementing a practical chatbot, I would like to research what applications chatbots are most suitable for and under which circumstances a chatbot is less beneficial.

-	What are chatbots?
-	What have people done with them?
-	What can they be used for?
-	Why would you use them?
-	Why are they so hyped?
-	How do you use them?
-	Which platform is best for me?
-	How to develop one?
-	What are they really good at?
-	What are they not good at?
-	What can we expect from the future?

\chapter{Defintion of the Term Chatbot}


Before working on a topic it is necessary to have common definitions of its vocabulary. This section will help aligning previous assumptions.
\\

The term \emph{chatbot} consists of two parts - \emph{chat} and \emph{bot}. The meaning can be better understood after examining the two components separately.
\\

The Oxford dictionary defines \textbf{chat} as ``an informal conversation'' and more specifically as ``the online exchange of messages in real time with one or more simultaneous users of a computer network''\cite{oxfordchat}.
\\
As apparent in this definition conversations play a central role in \emph{chats} and therefore \emph{chatbots}. But before examining what characterizes a conversation other noteworthy aspects of this definition are the inherent \emph{informal} format of a \emph{chat}, and the traits of being \emph{online} and \emph{real time}.
\\
Informality doesn't have to be seen as a strict requirement; however a chat massage and, for example, a classical letter have different degrees of formality.
\\
Being \emph{online} and thereby not bound to a specific location, device or other physicality can be seen as critical foundation for determining potential types of systems to interact with such media.
\\
The aspect of limiting communication to \emph{real time} gives an important restriction on possible interactions and sets a baseline for the expected user experience. This also excludes the usage of certain technologies that don't allow for the desired responsiveness.
\\
A \emph{conversation} is defined as ``a talk, especially an informal one, between two or more people, in which news and ideas are exchanged''\cite{oxfordconversation}.
\\
Important here is that there are always at least \emph{two} parties involved in communication and that information is \emph{exchanged}. Keeping that in mind the kind of systems involved in this should always receive and provide information; chatbots can not work with solely unidirectional interaction.
\\

\textbf{Bot} is defined as being ``(chiefly in science fiction) a robot'' with the specific characteristics of representing ``an autonomous program on a network (especially the Internet) which can interact with systems or users, especially one designed to behave like a player in some video games''\cite{oxfordbot},
\\
Foremost this provides the information that \emph{bots}, including \emph{chatbots}, are \emph{programs}. It's important to internalize that the creation of a chatbot is eventually the creation of an artifact in the form of a computer program.
\\
Furthermore the aspect of \emph{autonomy} and the communication over a \emph{network} can be connected with the previous described trait of a \emph{chat} being \emph{online}. The program is given autonomy by not being bound to any specific device. Building on this allows for different solutions than a scenario where the user is in full control of a programs behavior.
\\
Lastly there is a hint in this definition pointing out that a \emph{bot} can often be seen as a \emph{player} in a \emph{game}. This trend towards game-like mechanisms and the previous mentioned \emph{informality} suggest the utilization of playful interactions.
\\

Concluding from the combination of these definitions a chatbot can be defined as an autonomous computer program that interacts with users or systems online and in real time in the form of, often play-like and informal, conversations.

About Conversational Interfaces


Having defined the concept and character of chatbots, one apparent attribute of the technology under examination is that its domain of application involves interaction with one or more users.

Any form of interaction requires an interface as a way to interact.
An **interface** is generally described as a *"point where two systems, subjects, organizations, etc. meet and interact"* and in the area of computing it can be further defined as a *"device or program enabling a user to communicate with a computer"*. [^1]

This definition leaves a broad array of possible manifestations.
However, the range of suitable means of communication can be further narrowed by including another aspect of the previous definition, namely that interaction happens in the form of conversation.

Many communication mechanisms can be excluded by focusing on this characteristic of the interaction.
<br>
The term *conversation* strongly suggests the usage of natural, human language as a means of interaction while discouraging the idea of using an interface consisting purely of static or graphical elements.
<br>
Picturing a *conversational* interface one might expect a flow of interactions consisting of elements from all involved parties similar to the interfaces humans use to interact with each other.
<br>
It should also be noted that conversations are not limited to written language. Spoken language is also a possible interface for communication.
<br>

Focusing on written, text-based communication, it becomes apparent that not all text-based interfaces fit the characteristics of a *conversational interface*.
<br>
Classical command line interfaces often used to interact with computers are one example of text-based interfaces which don't have the attributes of conversational interfaces.
As implied in the naming, these interfaces use *commands* for interaction.
A command is an *"authoritative order"* [^2] which contrasts with a conversation.
<br>
The term *conversation* has a connotation of complex, non-linear communication where each involved party understands the underlaying ideas communicated opposed to merely receiving the characters the words consist of.

This deeper understanding of the intends a user has and the ability to adjust the interaction with a conversation consisting of customized parts sets conversational interfaces, and thus chatbots, apart from other interfaces.



[^1]: https://en.oxforddictionaries.com/definition/interface
[^2]: https://en.oxforddictionaries.com/definition/command

\chapter{Applications}


The following chapters explore different applications of chatbots.
\\
Starting with past work in the first chapter and looking at the present day in the subsequent chapters,
different approaches and products are presented.  Apart from highlighting interesting products, whole categories with potential benefits of using chatbots are defined and equally, problematic areas are identified as well.
\\
Furthermore the current ecosystem is portrayed; this includes involved companies, available target platforms and additional tooling.
\\
Moreover, current and potential users of chatbot products are analyzed and important questions about the current state of chatbot usage are addressed.
\\
Lastly the aspired goals, potentials and promises of chatbots are further identified.

\chapter{History and Past Work}


Before working on new projects with nowadays technology one should know about prior work and learn from past ideas - both, succeed and failed attempts.
\\
This chapter presents selected events from the last century, which introduced ideas forming the present definition of chatbots.
\\
It should be noted that this is no attempt at presenting an all-encompassing overview about the history of computing, instead the aim is to explain where the idea of chatbots and the interest of creating them originated from.


\section{1950 - The Turing Test}

Even before the term \emph{chatbot} was coined people started working on machines that interact with humans.
\\
One of the first important milestones was the 1950 paper "Computing Machinery and Intelligence" by Alan Turing\cite{turing}. His ideas back then are still an important building block of what is call a \emph{chatbot} in todays world and his thoughts are still central to many discussions about artificial intelligence.
\\
The most famous idea from this paper is the so called "Turing Test". This test is meant to identify if a machine posses human-like intelligence.
\\
Originally Turing described this test as "imitation game" whereas the experiment consist of a human interacting with two parties via \emph{textual messages}. One of the parties is another human and one is a machine. The human doesn't know upfront which one is the machine, but only that one of them will be a machine. During the \emph{game} the human can interact, via what we nowadays call \emph{chatting}, with the other party and is free to use any variation of messages. If the human is not able to tell which of the two parties is a machine and which one is a human, the machine passes the Turing Test.
\\
When creating a chatbot or another kind of artificial intelligence this test can still be applied to test the human-likeliness of the created machinery. Up to this point in time the Turing Test still remains to be challenged by new systems.


\section{1966 - ELIZA}

14 Years after the Turing Test was defined Joseph Weizenbaum started working on what would be known as the first program to pass a (limited) version of the Turing Test. Joseph Weizenbaum began working at MIT Artificial Intelligence Laboratory in 1964 and he released the ELIZA program in 1966.
\\

The original version of ELIZA was written in a programming programming language called MAD-Slip which was also created by Joseph Weizenbaum himself and it ran on the IBM 704 computer.
\\
ELIZA creates responses to natural language messages a user inputs via a text-based terminal.
\\

The most famous implementation of ELIZA is called DOCTOR and simulates a Rogerian psychotherapist. Rogerian psychotherapy is a person-centered therapy intended to let the client realize their own attitudes and behavior. Relying on mostly simple methods, it remains a popular treatment. Most answers the therapist gives are questions for further details about information which the client mentioned previously. Furthermore clients mostly keep the assumption that a therapist has specific intentions even when asking a non-obvious question.
\\

ELIZA takes advantage of the structure of the English language; the program takes apart sentences via pattern matching and keywords and reuses phrases after substituting certain words.
\\
For example, a clients answer ``Well, my boyfriend made me come here.'' can be transformed to ``Your boyfriend made you come here?''\cite{elizatest}.
\\
Certain signal words or sentences containing no signals words can be answered with generic, static phrases. Detecting the signal word ``alike'' in the sentence ``Men are all alike.'' ELIZA could pick the programmed phrase ``In what way?'' as answer\cite{elizatest}.
\\

Knowing about the nature of Rogerian psychotherapy, Joseph Weizenbaum created ELIZA initially intended as a parody showing off the simple behavior necessary to imitate this therapy.
\\
He was surprised that even people knowing about the inner workings of the program ended up having serious conversations with ELIZA. In one anecdote Joseph Weizenbaum tells how his secretary, after starting a conversation with ELIZA, asked him: ``Would you mind leaving the room, please?''\cite[5]{weizenbaum}.
\\

Lead by the success of the experiment he published the book "Computer power and human reason: from judgment to calculation" in 1976 which discusses his thoughts about artificial intelligence including the differences between machines and humans and the limits of computer intelligence.
\\
In the book he admits that he had not realized ``that extremely short exposures to a relatively simple computer program could induce powerful delusional thinking in quite normal people.''\cite{bbcnowthen}. This idea coined the term \emph{Eliza Effect} which describes people assuming computers to behave like humans. This term is still in use today.


\section{Further Noteworthy Developments}

Another famous program was published by the psychiatrist Kenneth Colby in 1972.
\\
He created PARRY as an attempt to simulate a human with paranoid schizophrenia. The implementation of PARRY is far more complex than ELIZA and it also models a personality including concepts of how to have conversations.
\\
The most famous demonstration of PARRY was at the first International Conference on Computer Communications (ICCC) in 1972 where PARRY and ELIZA had a conversation with each other.\cite{internethistory}
\\
Later on in scientific experiments PARRY also passed a version of the Turing Test.
\\

Further programs that have been created to pass the Turing Test and which gained the publics attention include
\\
\emph{Jabberwacky} which was started in 1988 and attempts to learn from the users input\cite{jabberwacky},
\\
\emph{Dr. Sbaitso} which was released in 1991 as an ELIZA-like demonstration for a sound card and was one of the first chatbots for MS-DOS based personal computers\cite{pcmag},
\\
and A.L.I.C.E., which has been first released in 1995 and became famous for its realistic behavior that is based on heuristic patterns instead of static rules\cite{approximatinglife}.
\\

The origin of the term \emph{chatbot} itself can be seen in a paper called ``ChatterBots, TinyMuds, and the Turing Test: Entering the Loebner Prize Competition'' published by Michael L. Mauldin in 1994, whereby \emph{chatbot} can be seen as a variation of the original version \emph{ChatterBots}.\cite{aiconf}
\\

Up until today the Turing Test has only been passed limited to certain domains and there is no chatbot yet that is able to simulate general human behavior indistinguishably from a real human.
\\
It needs to be noted that although creating an as human-like as possible system remains a popular challenge, not all applications of chatbots profit from this type of behavior. Many systems are instead optimized to provide quick and efficient interactions and behave accordingly without attempting to hide their artificiality.

\chapter{Present Day}


With more than sixty years of history the concept of chatbots is not new.
\\
People have been fascinated with the idea of being able to \emph{talk to computers} since a long time; but past attempts have mostly been simple experiments or applications focused on the aspect of entertainment associated with science fiction inspired machines.

Lately the technology industry and press are increasingly interested in the topic of conversational interfaces.
\\
The reinforced interest can be explained by observing recent developments of technology and recent market trends.

Since Apple's release of Siri in 2011\cite{iphonelaunch} customers have become more aware of the possibilities of conversational interfaces. Even though the capabilities were limited at that time, functionality improved quickly in the following years, which can be attributed to the new competition Apple triggered in the market.

Simultaneously, in this period of time artificial intelligence gained new traction due to the success of using neuronal networks for machine learning\cite{mltrend}.
\\
The concept of neural networks ``dates back to the 1950s, and many of the key algorithmic breakthroughs occurred in the 1980s and 1990s''\cite{airevolution}, but only now they are successfully applied. This is mainly due to the increased computing power available now. A second important condition is the amount of data available today; big Internet companies specialize on collecting data, originally intended to better target advertisement, but now they can use their data to train neuronal networks. Neuronal networks have their name because they are modeled after human brains; instead of specific rules of what the program should do the machine learns from examples in similar ways to how humans learn. Some task that are too complex to solve with a rule-based program, can now be solved by collecting enough example data and letting the machine figure out the solution instead.
\\
This techniques can also be applied in the field of natural language processing, which is essential to understanding and generating text for conversational interfaces.
\\

With these new technical possibilities more people see \emph{conversation as an interface} not only as an idea of science fiction movies but instead as something that could be possible in the real world.
\\

In addition to these new technical possibilities recent market trends make chatbots more compelling. ``Computing is rapidly shifting to mobile devices''\cite{mobileusage} and ``messaging apps have surpassed social networks in monthly active users''\cite{convtrends}. This development means that users don't have space for complex interfaces on the small screens of their devices, they need a solution which is light-weight in data consumption, and they can not use complicated keyboard shortcuts. At the same time users already spend the majority of their time in messaging applications and therefore, they are well accustomed to this way of communicating.
\\

Originating from the current state of technology the increased interest in conversational interfaces creates new platforms, products and applications for chatbots. These will be subjects of the subsequent chapters.




\chapter{Platforms}


Unless software is distributed on dedicated hardware software products are designed to be run and accessed through other software. The underlying software is the platform a product is created for.
\\
Products that target operating systems such as Microsoft Windows or Apple's iOS require users to install necessary executable files on their local system.
\\
Other software uses the web as platform, whereby customers use a web browser to access the software over the internet, while the software itself is executed on another computer called \emph{server}.
\\

In the case of chatbots the target platform can be any medium which enables users to send messages to each other. A chatbot can be seen as a counterpart to interact with in the same way users interact with other humans.
\\

There are numerous platforms available that fulfill these requirements.
\\

It should be noted that while messaging platforms provide means of communication, chatbots function similar to software accessible via a web browser; a server executing the chatbot software is needed and the messaging platform communicates with the server in the same way a web browser does communicate with a server on the users behalf.
\\

Because of the wide variety of available messaging platforms it is not possible to create an all-encompassing collection of available platforms in the context of this work; the following is an overview over the currently most popular platforms, their capabilities and their area of focus.
\\

To begin with, one of the most used online communication platforms is E-Mail.
\\
E-Mail however does not provide the earlier defined characteristics of chatbots to be able to communicate \emph{informally} and in \emph{real time}; which disqualify E-Mail as a platform for chatbots, even though it can be automated in similar ways.
\\

A well-known communication technology, which is suited for chatbots, is \emph{Short Message Service}, short SMS.
\\
SMS is primarily used on mobile devices and users are identified by their phone numbers, wherefore the communication has to happen through cell network providers. The technology is limited in number of characters, often users are charged by amount of messages sent and communication is limited to text-only. ``End of year 2009 user level for SMS globally was 78\%, ie 3.6 Billion''\cite{mobilenumbers} people worldwide which means it remains one of the most popular communication channels; and therefore is an interesting option for applications requiring a low entry barrier.
\\

Since chatbots can communicate not only via text but also using voice, \emph{phone calls} are also a possible medium. They are a common way of communication available to a large number of people. However, relying solely on voice for communication without any visual feedback the design of the user experience has to be thought out especially careful. Furthermore not only being able to understand and generate natural language, but also be able to parse and generate voice comes with further development cost.
\\

Apple's \emph{Siri} is another voice based system available; but it is not accessible as platform for external services.
\\
Voice based systems that can be targeted as platforms are Amazon's \emph{Alexa} and \emph{Google Assistant} Both systems are general assistants helping the user with a variety of task and in both cases they allow delegating specific tasks to third parties so they can handle the task at hand.
\\

Currently popular target platforms for chatbots are messenger platforms. They are primarily text based, they mostly come without cost for the end user and additional to text they often support multimedia formats such as pictures, audio, locations and stickers. Some platforms also allow developers to display sliders, buttons and other graphical interface elements to the user, which can help guiding users instead of exclusively relying on natural language for communication.
\\

Facebook's \emph{WhatsApp} is with one billion active users in January 2017\cite{fbpopular} one of the most popular messenger applications. It, however, does not allow automated access to its platform and therefore using it as a chatbot platform is not a viable option.
\\

The second messenger application belonging to Facebook, called \emph{Facebook Messenger}, is equally popular with one billion active users in January 2017\cite{fbpopular} as well. Contrary to WhatsApp, Facebook Messenger provides a platform for developing chatbots. Counting over 100,000 monthly active chatbots\cite{messenger} it is an interesting platforms to develop for.
\\

Following in popularity\cite{appusage} are two Chinese messenger platforms, \emph{QQ Mobile} and \emph{WeChat}. Both of them currently do not provide a specific chatbot platform, but there have been successful attempts at creating chatbots for these platforms\cite{wechatbot}.
\\

Further popular messenger applications are the Japanese \emph{Line} Microsoft's \emph{Skype} \emph{Telegram} and the more business focused \emph{Slack} All of these applications also provide dedicated platforms for the development of chatbots.
\\

The choice of platform mostly depends on the target market. Different audiences prefer different platforms and a product might be better suited for certain environments.
\\
As with the creation of other software it is also possible to release the same chatbot software for multiple targets. In similar ways to other platforms, the interaction of the software with the platform has to conform to the technical details and protocols of each target, the usage of platform-specific features has to be adapted individually and the user experience needs to be designed to fit each environment's expectations.

\chapter{Products}


With the amount of messaging platforms opening up for chatbot development, companies have become interested in releasing their product for this new format and some developers created new products focusing solely on the chatbot market.
\\

Chatbot development is still a new market with a lot of changes happening all the time but there are some current trends in what companies are interested in creating. The direction might change soon in the future and there are probably still many undiscovered possibilities.
\\

One helpful classification of chatbots is categorizing them in terms of features they provide. The following categories are adapted from the article ``7 Types of Bots'' by \emph{Dotan Elharrar, a Product Manager at Microsoft AI \& Research}\cite{bottypes}.


\paragraph{Single-feature Chatbots}

A popular category of bots provide only one single feature. These bots are limited in functionality but simple to use. One example is a Facebook bot called Instant Translator\cite{instanttranslator}; in the beginning the user selects one language to translate to and all the bot does from there on is to translate all text it receives from the users language to the selected target language.


\paragraph{Proactive Chatbots}

This category are chatbots that push information to the user instead of answering questions in conversations. Hereby the user doesn't need to interact with the chatbot, but only uses it as service to receive information at certain times. One example would a service which sends the user a daily weather forcast. Another use-case is the Chatbot from the airline KLM\cite{klm}; a user can use the service to get updates and information about their flights delivered.


\paragraph{Group Chatbots}

There is a range of functionality chatbots can provide when they interact with a whole group of people instead of only a single user. These chatbots are limited to platforms which provide the necessary features to use chatbots in group conversations. A simple example for a Group Chatbot is called Roll\cite{venturebeat}; when sending a question to Roll the chatbots answers with one name of the members of the group.


\paragraph{Simplification Chatbots}

In a few cases chatbots are used to provide users with a simpler interface to complicated existing tasks which traditionally require the handling of a lot of bureaucracy. One example is a service called DoNotPay. It is advertised as "the world’s first robot lawyer"\cite{oreilly} and the service helps the user with simple legal problems, such as fighting a parking ticket.


\paragraph{Entertainment Chatbots}

One of the most popular kind of chatbots are still chatbots whose functionality consists only of having conversations with users. These services don't interact with other resources apart from the conversation itself. The previous described ELIZA belongs to this category.


\paragraph{Personal Assistants}

This category consists of chatbots that combine many different features and can be seen as platforms of their own. Siri and Alex, which have been mentioned earlier, belong to this category.


\paragraph{Optimization Chatbots}

Most companies are interested in chatbots in this category. The idea is to make an existing product more accessible by creating a chabot for users to connect to the product. By taking advantage of new platforms companies like to reduce friction for customers to use the product. The currently most obvious aspect of the chatbot platforms is the ease for users to access the products. Companies like to optimize the use of their product by making it available via the conversational interfaces of chatbots.
\\

Use cases fitting this category can be found across many different industries. The article ``100 Best Bots For Brands \& Businesses''\cite{topbots} lists examples from different industries using chatbots to optimize access to their products. Products include beauty brands such as Sephora, consumer goods like Johnnie Walker, entertainment companies including Disney and Marvel, fashion brand such as H\&M, financial services like PayPal, food delivery from stores such as Pizza Hut, E-Commerce platforms including eBay, Traveling services such as Airbnb and Expedia, Airlines like Lufthansa and British Airways and many news outlets including Washington Post, New York Times, Forbes and BCC.
\\


As apparent from the engagement of many well established companies, brands are very interested in being present on messenger platforms.
\\

While there is big interest in targeting messaging platforms as a new market, the consumer engagement and the development efforts are nothing compared to established markets such as mobile applications or the web. Most of current products only provide limited features and are mostly created to connect customers to existing products.
\\

With increasing interest, engagement and, subsequently, financial investments the arise of more sophisticated products can be expected in the future.

% \input{chapters/09-innovative-ideas.tex}
% \input{chapters/10-companies.tex}
% \input{chapters/11-usage.tex}
\input{chapters/12-potentials-and-promises.tex}
% \input{chapters/13-readme.tex}
% \section{Choosing a Practical Example}


A suitable application for a chatbot needs to be selected before thinking about implementation details.
\\
As illustrated earlier, chatbots can be used to cater a wide variety of applications.
\\

Since this application should be a demonstration of the different aspects of developing chatbots,
it should not be too simplistic in scope.
An appropriate example covers more than one of the product categories described in section \ref{classification} on page \pageref{classification},
while being, at the same time, not too technical challenging in a specific  problem domain outside of chatbot development.
\\
A service, that accepts image files and returns the same picture with a filter applied,
might be an interesting and entertaining use case for a chatbot,
however, it would not be an adequate example to give a general introduction to chatbot development since,
even though it would be a technical interesting task to solve regarding image processing,
it would not illustrate much technical details in the domain of chatbot development.
\\

The here selected example application is a system for \textbf{individual language studying}.
\\
While this idea arises from a personal need for such a system and an observed shortage in the functionalities existing solutions provide,
this application also fulfills the stated requirements of a suitable example application.
\\

The language studying chatbot covers multiple of the categories defined in section \ref{classification}.
\\
It is mainly an \emph{optimization chatbot} improving the task of language studying,
but the chatbot can also be classified as a \emph{proactive chatbot} because it can use proactive features to notify the user when it is time for studying.
\\

Although there is a necessity to understand parts of the problem domain of language studying,
the required domain-specific knowledge is minimal.
\\

The idea is to build a system independent from existing learning resources that users can use to study their own vocabulary.
Since there is no need to focus on content for specific languages, the main focus remains implementing the chatbot features,
which are primarily, that a user is able to input new vocabulary, and then the system tests the user's knowledge in appropriate studying intervals.

% \section{Existing Solutions}
\label{existing}


When starting a new projects it is helpful to research for existing solutions that might solve the same problem.
\\

There is already a variety of existing software applications for language studying, which have very different use cases and solve different problems.
\\

One significant separation is between software that includes content and software users can customize to study personal content.
\\

The first segment is the most prominent.
This software is intended to enable people to self-study language and, at least partially, replace physical language courses.
\\
A main reason for the prominence of this segment can be attributed to the ability to sell content.
Professionally curating the curriculum for a language course requires teaching expertise and is a lot of effort. Because of this content remains expensive, not only in software but also in the form of physical textbooks.
\\
One popular example from this segment is \emph{Duolingo}. ``Duolingo has courses in a handful of languages.... The courses are structured in a way like games as well - you earn skill points as you complete lessons''~\cite{lifehacker}.
\\
Interestingly, \emph{Duolingo} recently released a chatbot~\cite{topbots2} as part of their \emph{iPhone} application which enables the user to have a written conversation about certain topics with a chatbot to learn the appropriate phrases for the given scenario.
Although the topics and possible phrases are restricted in each scenario, this is a first example of how chatbots can be used for language studying.
\\

The second segment consists of software which does not provide users a guideline what to study but instead is intended to support users studying their own content.
\\
Most of the software that can be found in this segment are attempts to bring traditional flashcards to digital media.
\\

One of the most established applications in this segment is \emph{Anki}~\cite{lifehacker}, which exists for more than ten years already and provides a flexible, but also rather complex, interface to create a various kinds of studying material.
\\
Another more recent competitor is \emph{Memrise}, where users get a more intuitive interface, which also includes several gamification\footnote{The Oxford Dictionaries describe gamification as ``the application of typical elements of game playing (e.g. point scoring, competition with others, rules of play) to other areas of activity''~\cite{oxfordgamification}} features to make the studying process more appealing.
\\

Both mentioned products are not restricted to a field of study and users are able to add their own content.
Furthermore both products also offer mechanisms for users to share content with others, which allows users to reuse what other users created.
\\

For the here planned chatbot example the second segment is more fitting, since there are no resources in the current context to curate professional content.
\\
The following implementation is an attempt at creating a software product with a conversational text interface that is in its use cases similar to products like \emph{Anki} and \emph{Duolingo} while making use of the unique features the medium chatbot provides.

% \section{Use Cases and Requirements}


Building on the analysis of existing solutions in \ref{existing} on page \pageref{existing}, features for the chatbot need to be specified.
An effective method for gathering crucial features is to find potential users
and create usage scenarios for their individual needs.
\\

To apply this method, the fundamental problem the application is solving needs to be defined first.
\\
The issue this chatbot is trying to help with is the study of individual vocabulary.
The goal is not to provide studying material in a way a language course or a textbook does,
but instead to complement these resources with a tool to study new vocabulary and phrases students pick up
while studying or in different situations in every day life.
\\


\subsection{User Stories}

The following are two hypothetic scenarios of individuals that might use the chatbot and both profit from it in different ways.
\\

Clara is a 22 years old American.
She moved to New York City to go to university.
Currently she is in the last year of her bachelor degree in economics.
In university she signed up for an evening class in Mandarin.
She uses \emph{Facebook Messenger} every day to talk to her friends and she discovered the chatbot when a friend sent her a link.
For her the most difficult part of the studies is to write hànzì, the Chinese characters.
Now Clara uses the chatbot to write down vocabulary in hànzì during her class,
and at home she revises the new characters by going through them using the chatbot and writing the characters
down on paper.
\\

Pierre is 29 years and born in Bordeaux, France.
He studied computer science and a year ago he moved to Berlin where he found a job in a startup.
At work all communication is done in English since the team consists of people from all around the globe.
Because Pierre is not a native English speaker, he picks up new words at work almost every single day.
Since moving to Berlin Pierre also made a few German friends and he tries to pick up new words they teach him.
He found the chatbot on a news website for technology products,
and since then whenever Pierre learns a new word he grabs his phone from his pocket and adds the word to the chatbot.
Since the chatbot has no restrictions on what to learn, Pierre uses it to save both, German and English, vocabulary in one place.
Pierre's daily commute from and to work takes 40 minutes twice a day.
Now he uses his commute time to take out his phone and review new vocabulary he picked up the previous days.
\\


\subsection{Functional Requirements}
\label{funcreq}

All necessary functional requirements can be extracted from the above defined user stories.
\\
First, a user needs to be able to add new vocabulary.
There should not be any restrictions on what can be added
and vocabulary should not be limited to single words, because in many cases it is more helpful
to add whole phrases instead.
Each vocabulary consists of the phrase the user tries to memorize
and an explanation to help understanding the meaning of the phrase.
\\
Next, the chatbot should provide a way to revise vocabulary.
There should be two possible modes for revising;
one where users can click a button to tell whether they remembered the phrase correctly or not,
and a second mode whereby users type out the phrase themselves.
In each case the system should keep track of whether users knew the correct solution or not.
\\
Lastly, it is necessary to determine what to study next.
A user should not be required to think about what or when to review vocabulary.
The chatbot needs a system to decide the review time for each vocabulary,
and ideally the user is notified when vocabulary is ready to be reviewed by sending a message to the user.
\\
These three main features can be seen as a sufficient \emph{minimal viable product}, or \emph{MVP}.
\\

For demonstration purposes it is desired to keep the product as simple as possible.
The knowledge that can be taken from making decisions about the implementation and walking through the process of creating the chatbot,
is mostly independent from this particular product and can be applied to the development of other chatbot products.


\subsection{Non-functional Requirements}

Since this is a simple example, non-functional requirements remain minimal.
\\
\emph{Availability} of the service is not a priority, but chatbot software can be scaled similar to other software,
and redundancy can be used to ensure availability.
Since messaging platforms act as intermediary between users and the chatbot software, most platforms also re-send missed messages in case the chatbot is unavailable.
That the platform ensures availability, further lessens the priority to address it in the chatbot software itself.
\\
Similarly, \emph{security} is not a main focus here, because the messaging platform itself already handles certain security-sensitive functionality such as authentication and encryption of communication. A production scenario, though, would require further care for securing the service.
\\
\emph{Performance} is equally not a major concern.
Because the scope of the example application is limited,
the domain specific logic remains inexpensive in computation.
The main performance bottleneck is the in \ref{limitations} on page \pageref{limitations} mentioned aspect of networking and involved unknown parties.
Employing performance-improving solutions for networking issues won't be a part of the example chatbot,
but performance can be improved by choosing geographically strategic located data centers for deploying the chatbot software.
\\
A more central requirement is \emph{reusability}.
Although the example focuses on solving a specific task,
the software architecture should be designed in a way,
that appropriate parts can be reused for other chatbots in the future.
To ensure reusability the software should be \emph{documented}, \emph{stable} and \emph{extensible}.
\\
\emph{Usability} can be seen as the most important non-functional requirement.
The focus of developing the example chatbot is to design a good user experience and to explore how interface and interaction design can be best accomplished with the given medium.

% \input{chapters/17-platform-evaluation.tex}
% \input{chapters/18-setup-overview.tex}
% \section{Feature Implementation}

-	Presentation of the product
-	Show some screenshots for usage of features
-	Explain how to use features
-	Explain how each feature is implemented


Could add subsection about testing here.
How to test a bot seems kind of tricky.


\subsection{Architecture}

- architecture (+ Graphic)

% \input{chapters/20-technical-details.tex}
% \input{chapters/21-obstacles-and-limitations.tex}
% \input{chapters/22-conclusions.tex}
% \input{chapters/23-outlook-and-possibilities.tex}
% \input{chapters/24-references.tex}


\printbibliography[heading=bibintoc]


\chapter*{Erklärung}

Hiermit versichere ich, dass ich die vorliegende Arbeit selbstständig verfasst und keine anderen als die angegebenen Quellen und Hilfsmittel benutzt habe, dass alle Stellen der Arbeit, die wörtlich oder sinngemäß aus anderen Quellen übernommen wurden, als solche kenntlich gemacht und dass die Arbeit in gleicher oder ähnlicher Form noch keiner Prüfungsbehörde vorgelegt wurde.

\vspace{3cm}
Ort, Datum \hspace{5cm} Unterschrift\\

\end{document}
